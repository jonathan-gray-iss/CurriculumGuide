\theoremstyle{plain}
\newtheorem{thm}{Theorem}%[section]
\newtheorem*{thmn}{Theorem}
\newtheorem*{conj}{Conjecture}
\newtheorem{prop}[thm]{Proposition}
\newtheorem{lemma}[thm]{Lemma}
\newtheorem{cor}[thm]{Corollary}
\newtheorem*{postn}{Postulate}
\newtheorem{const}{Construction}
\newtheorem*{quesn}{Question}
\newtheorem*{quesns}{Questions}
\newtheorem*{prob}{Problem}

\theoremstyle{definition}
\newtheorem{defn}[thm]{Definition}
\newtheorem{ex}[thm]{Example}
\newtheorem{hyp}[thm]{Hypothesis}
\newtheorem{ques}[thm]{Question}

\theoremstyle{remark}
\newtheorem{rem}{Remark}
\newtheorem*{remn}{Remark}
\newtheorem{rrule}[thm]{Rule}
\newtheorem*{rrulen}{Rule}
\newtheorem*{notn}{Note}







%%%%%%%%%%
% blue box with title bold and in-line
%%%%%%%%%%


%\newenvironment{bluebox}[1][]{%
%\begin{mdframed}[%
%	skipabove=\baselineskip plus 2pt minus 1pt,
%    skipbelow=\baselineskip plus 2pt minus 1pt,
%innertopmargin=4pt,
%innerleftmargin=4pt,
%innerrightmargin=4pt,
%linewidth=0.0pt,
%linecolor=blue!15,
%backgroundcolor=blue!15,
%userdefinedwidth=10.5cm,
%align=center,
%roundcorner=4pt
%]\noindent\textbf{#1}\qquad%
%}{%
%    \end{mdframed}
%}
%
%
%
%%%%%%%%%%%
%% blue box with title bold and separate line
%%%%%%%%%%%
%
%
%\newenvironment{titlebox}[1][]{%
%    \begin{mdframed}[%
%        frametitle={\large #1},
%        skipabove=\baselineskip plus 2pt minus 1pt,
%        skipbelow=\baselineskip plus 2pt minus 1pt,
%        linewidth=0.5pt,
%linecolor=blue!15,
%        frametitlebackgroundcolor=blue!15,
%backgroundcolor=blue!15,
%userdefinedwidth=10.5cm,
%align=center,
%roundcorner=4pt
%    ]%
%}{%
%    \end{mdframed}
%}


%%%%%%%%%%
% wide blue box with title bold and separate line
%%%%%%%%%%


%\newenvironment{widetitlebox}[1][]{%
%    \begin{mdframed}[%
%        frametitle={\large #1},
%        skipabove=\baselineskip plus 2pt minus 1pt,
%        skipbelow=\baselineskip plus 2pt minus 1pt,
%        linewidth=0.5pt,
%linecolor=blue!15,
%        frametitlebackgroundcolor=blue!15,
%backgroundcolor=blue!15,
%userdefinedwidth=14.5cm,
%align=center,
%roundcorner=4pt
%    ]%
%}{%
%    \end{mdframed}
%}
%
%
%%%%%%%%%%%
%% Exercises Heading
%%%%%%%%%%%
%
%\newcommand{\exercises}{{\color{halfgray}\exertitle \noindent Exercises\,\,}{\exercs\color{halfgray}\arabic{chapter}.\arabic{section}} {\color{blue!15}\leaders\hrule height .8ex depth \dimexpr-.8ex+0.8pt\relax\hfill
%\vspace{3mm}} }
%
%%%%%%%%%%%
%% fancy example style
%%%%%%%%%%%
%
%
%\theoremstyle{definition}
%\newtheorem{exinn}{Example}[chapter]
%\newenvironment{example}
%  {\clubpenalty=10000
%   \begin{exinn}%
%   \mbox{}%
%   {\color{blue!15}\leaders\hrule height .8ex depth \dimexpr-.8ex+0.8pt\relax\hfill}%
%   \mbox{}\linebreak\ignorespaces}
%  {\par\kern2ex\color{blue!15}\hrule\end{exinn}}
%
%

%%%%%%%%%%
% On-Your-Own example 
%%%%%%%%%%
%
%\newtheoremstyle{noperiod}% name
%  {}%      Space above
%  {}%      Space below
%  {}%         Body font
%  {}%         Indent amount (empty = no indent, \parindent = para indent)
%  {\bfseries}% Thm head font
%  {}%        Punctuation after thm head
%  {.5em}%     Space after thm head: " " = normal interword space;
%        %       \newline = linebreak
%  {}%         Thm head spec (can be left empty, meaning `normal')
%
%\theoremstyle{noperiod}
%\newtheorem*{noperiod}{On Your Own \ldots}
%\newenvironment{oyoe}
%  {\clubpenalty=10000
%   \begin{noperiod}%
%   \mbox{}%
%   {\color{blue!15}\leaders\hrule height .8ex depth \dimexpr-.8ex+0.8pt\relax\hfill}%
%   \mbox{}\linebreak\ignorespaces}
%  {\par\kern2ex\color{blue!15}\hrule\end{noperiod}}


%%%%%%%%%%
% Custom commands
%%%%%%%%%%



\newcommand{\norm}[1]{\left\Vert#1\right\Vert}
\newcommand{\abs}[1]{\left\vert#1\right\vert}
\newcommand{\ip}[1]{\left\langle#1\right\rangle}
\newcommand{\set}[1]{\left\{#1\right\}}
\newcommand{\divides}{|}
\newcommand{\notdivides}{\!\!\not\divides \,}

% Standard sets

\newcommand{\Real}{\mathbb R}
\newcommand{\Int}{\mathbb Z}
\newcommand{\Nat}{\mathbb N}
\newcommand{\Rat}{\mathbb Q}
\newcommand{\Com}{\mathbb C}




\newcommand{\am}{_A\underline{\mbox{Mod}}}
\newcommand{\bm}{_B\underline{\mbox{Mod}}}
\newcommand{\bmo}{_{B^\circ}\underline{\mbox{Mod}}}
\newcommand{\To}{\longrightarrow}
\newcommand{\Hom}{\mbox{Hom}}
\newcommand{\ra}{\rightarrow}
\newcommand{\bds}{\begin{doublespace}}
\newcommand{\eds}{\end{doublespace}}
\newcommand{\mca}{\mathcal{A}}
\newcommand{\mcb}{\mathcal{B}}
\newcommand{\mcc}{\mathcal{C}}
\renewcommand{\phi}{\varphi}

\newcommand{\inv}[1]{{#1}^{-1}}
\newcommand{\da}[1]{(dx_{#1})_a}
\newcommand{\di}{\partial_i}
\renewcommand{\dj}{\partial_j}
\newcommand{\xTo}[1]{\xrightarrow{#1}}
\newcommand{\xMapsto}[1]{\xmapsto{#1}}
\newcommand{\dk}{\partial_k}
\newcommand{\bdy}{\partial}
\newcommand{\lcm}{\mathrm{lcm}}
\newcommand{\pstr}[1]{{\bf p}[#1]}
\newcommand{\psp}{\bf{p}}
\newcommand{\Aut}{\mathbb{A}}
\newcommand{\plo}{p\mathcal{L}}
\newcommand{\pls}{p\mathbb{S}}
\newcommand{\ls}{\mathbb{S}}
\newcommand{\cat}[1]{\textbf{\mbox{#1}}}

% high-light text, lists

%\newcommand{\hl}[1]{\colorbox{blue!15}{#1}}







\newcommand{\be}{\begin{enumerate}}
\newcommand{\ee}{\end{enumerate}}

\newcommand{\bal}{\begin{enumerate}[label=\textbf{\alph*.}]}
\newcommand{\eal}{\end{enumerate}}

\newcommand{\bexer}{\begin{enumerate}[label=\textbf{\large\arabic*.}\,\,]\itemsep=5mm}
\newcommand{\eexer}{\end{enumerate}}

\newcommand{\btc}{\begin{multicols}{2}}
\newcommand{\etc}{\end{multicols}}

\newcommand{\bmc}[1]{\begin{multicols}{#1}}

\newcommand{\dirns}[1]{\noindent \emph{#1}}

% Calculus

\newcommand{\deriv}[2]{\frac{\partial#1}{\partial x_#2}}
\newcommand{\vd}[2]{\frac{\partial#1}{\partial#2}}
\newcommand{\ddx}[1]{\frac{\partial#1}{\partial x}}
\newcommand{\ddy}[1]{\frac{\partial#1}{\partial y}}
\newcommand{\ddt}[1]{\frac{d#1}{dt}}

\newcommand{\Q}[1]{Q_{#1}^{(r)}}
\newcommand{\F}[1]{F_{#1}S\mathbb{A}_{r}}
\newcommand{\Out}{Out(F_n)}


% Canonical basis for R^3

\newcommand{\ivect}{\hat{\imath}}
\newcommand{\jvect}{\hat{\jmath}}
\newcommand{\kvect}{\hat{k}}



% Geometry

\newcommand{\ray}[1]{\overrightarrow{#1}}
\newcommand{\geoline}[1]{\overleftrightarrow{#1}}
\newcommand{\lineseg}[1]{\overline{#1}}
\newcommand{\arc}[1]{\wideparen{#1}}



%%%%%
%%  Number line tricks in tikz.  From User "Altermundus" at tex.se
%%%%%

%%%%%%%%%
%%%%%%%%%       Uncomment this for color version
%%%%%%%%%
%\newcommand{\mynumberline}[3]{%
%\begin{tikzpicture}[out=45,in=135,relative,>=stealth]
%\draw[<->] (#1-1,0)--(#2+1,0);
%\foreach \x in {\number\numexpr#1\relax,...,\number\numexpr#2\relax}  
%\draw[shift={(\x,0)},color=black] (0pt,2pt) -- (0pt,-2pt) node[below] {\footnotesize $\x$};
%
%\pgfmathsetmacro{\End}{#2-1} 
% \foreach \i in {#1,...,\End}{%
%    (\i,0) to (\i+1,0)
%} ; 
%\end{tikzpicture}}
%
%\newcommand{\mynumberray}[3]{%
%\begin{tikzpicture}[out=45,in=135,relative,>=stealth]
%\draw[->] (#1,0)--(#2+1,0);
%\foreach \x in {\number\numexpr#1\relax,...,\number\numexpr#2\relax}  
%\draw[shift={(\x,0)},color=black] (0pt,2pt) -- (0pt,-2pt) node[below] {\footnotesize $\x$};
%
%\pgfmathsetmacro{\End}{#2-1} 
% \foreach \i in {#1,...,\End}{%
%    (\i,0) to (\i+1,0)
%} ; 
%\end{tikzpicture}}
%
%\newcommand{\mynumberlinewithpoint}[3]{%
%\begin{tikzpicture}[out=45,in=135,relative,>=stealth]
%\draw[<->] (#1-1,0)--(#2+1,0);
%\foreach \x in {#3}
%\fill [color=black] (\x,0) circle (2.25pt);
%\foreach \x in {\number\numexpr#1\relax,...,\number\numexpr#2\relax}  
%\draw[shift={(\x,0)},color=black] (0pt,2pt) -- (0pt,-2pt) node[below] {\footnotesize $\x$};
%
%\pgfmathsetmacro{\End}{#2-1} 
% \foreach \i in {#1,...,\End}{%
%    (\i,0) to (\i+1,0)
%} ; 
%\end{tikzpicture}}
%
%
%
%
%
%
%\newcommand{\mysquarecoordplane}[2]{%
%\begin{tikzpicture}[scale=.5, out=45,in=135,relative,>=stealth]
%\draw[<->] (#1-1,0)--(#2+1,0);
%\foreach \x in {\number\numexpr#1\relax,...,-1}  
%\draw[shift={(\x,0)},color=black] (0pt,2pt) -- (0pt,-2pt) node[below] {\tiny $\x$};
%
%\foreach \x in {1,...,\number\numexpr#2\relax}  
%\draw[shift={(\x,0)},color=black] (0pt,2pt) -- (0pt,-2pt) node[below] {\tiny $\x$};
%
%\pgfmathsetmacro{\End}{#2-1} 
% \foreach \i in {#1,...,\End}{%
%    (\i,0) to (\i+1,0)
%} ; 
%
%\draw[<->] (0,#1-1)--(0,#2+1);
%\foreach \x in {\number\numexpr#1\relax,...,-1}  
%\draw[shift={(0,\x)},color=black] (2pt,0pt) -- (-2pt,0pt) node[right] {\tiny $\x$};
%\foreach \x in {1,...,\number\numexpr#2\relax}  
%\draw[shift={(0,\x)},color=black] (2pt,0pt) -- (-2pt,0pt) node[right] {\tiny $\x$};
%\pgfmathsetmacro{\End}{#2-1} 
% \foreach \i in {#1,...,\End}{%
%    (0,\i) to (0,\i+1)
%} ;
%
%\end{tikzpicture}}
%
%
%
%
%
%
%\newcommand{\addsubnumlinetoright}[3]{%
%\begin{tikzpicture}[out=45,in=135,relative,>=stealth]
%\draw[<->] (#1-2,0)--(#2+2,0);
%\foreach \x in {\number\numexpr#1-1\relax,...,\number\numexpr#2+1\relax}  
%\draw[shift={(\x,0)},color=black] (0pt,2pt) -- (0pt,-2pt) node[below] {\footnotesize $\x$};
%\fill (#1,0) circle (2pt);
%\fill (#2,0) circle (2pt);
%
%\pgfmathsetmacro{\End}{#2-1} 
% \draw[#3,shorten >=2pt]
% \foreach \i in {#1,...,\End}{%
%    (\i,0) to (\i+1,0)
%} ; 
%\node[color=Red] at (#2,-0.75) {\small End};
%\node[color=Green] at (#1,-0.75) {\small Start};
% \pgfmathsetmacro{\xtxt}{(#1+#2)/2}
%\node at (\xtxt,0.5) {\small Hop \number\numexpr#2-#1\relax\ units to the \emph{right}};
%\end{tikzpicture}}
%
%
%
%
%\newcommand{\symmnumbline}[3]{%
%\begin{tikzpicture}[out=45,in=135,relative,>=stealth]
%\draw[<->] (#1-2,0)--(#2+2,0);
%\foreach \x in {\number\numexpr#1-1\relax,...,\number\numexpr#2+1\relax}  
%\draw[shift={(\x,0)},color=black] (0pt,2pt) -- (0pt,-2pt) node[below] {\footnotesize $\x$};
%%\fill (#1,0) circle (2pt);
%%\fill (#2,0) circle (2pt);
%
%\pgfmathsetmacro{\Start}{#1-1} 
% \draw[#3,shorten >=2pt]
% \foreach \i in {\Start,...,-1}{%
%    (\i,0) to (-\i,0)
%} ; 
%
%\end{tikzpicture}}
%
%
%
%\newcommand{\disttozeronl}[3]{%
%\begin{tikzpicture}[out=45,in=135,relative,>=stealth]
%\draw[<->] (-3,0)--(3,0);
%\foreach \x in {\number\numexpr-3\relax,...,\number\numexpr3\relax}  
%\draw[shift={(\x,0)},color=black] (0pt,2pt) -- (0pt,-2pt) node[below] {\footnotesize $\x$};
%%\fill (1,0) circle (2pt);
%\fill (0,0) circle (2pt);
%
%\pgfmathsetmacro{\End}{#1+#2-1} 
% \draw[#3,shorten >=0pt]
% \foreach \i in {1}{%
%    (\i-1,0) to (\i,0)
%} ; 
%\end{tikzpicture}}
%
%
%\newcommand{\myaddsubnumlinetoright}[3]{%
%\begin{tikzpicture}[out=45,in=135,relative,>=stealth]
%\draw[<->] (#1-2,0)--(#1+#2+2,0);
%\foreach \x in {\number\numexpr#1-1\relax,...,\number\numexpr#1+#2+1\relax}  
%\draw[shift={(\x,0)},color=black] (0pt,2pt) -- (0pt,-2pt) node[below] {\footnotesize $\x$};
%\fill (#1,0) circle (2pt);
%\fill (#1+#2,0) circle (2pt);
%
%\pgfmathsetmacro{\End}{#1+#2-1} 
% \draw[#3,shorten >=2pt]
% \foreach \i in {#1,...,\End}{%
%    (\i,0) to (\i+1,0)
%} ; 
%\node[color=Red] at (#1+#2,-0.75) {\small End};
%\node[color=Green] at (#1,-0.75) {\small Start};
% \pgfmathsetmacro{\xtxt}{(#1+#1+#2)/2}
%\node at (\xtxt,0.5) {\small Hop \number\numexpr#2\relax\ units to the \emph{right}};
%\end{tikzpicture}}
%
%
%\newcommand{\addsubnumlinetoleft}[3]{%
%\begin{tikzpicture}[out=135,in=45,>=stealth]
%\draw[<->] (#2-2,0)--(#1+2,0);
%\foreach \x in {\number\numexpr#2-1\relax,...,\number\numexpr#1+1\relax}
%\draw[shift={(\x,0)},color=black] (0pt,2pt) -- (0pt,-2pt) node[below] {\footnotesize $\x$};
%\fill (#1,0) circle (2pt);
%\fill (#2,0) circle (2pt);
%
%\pgfmathsetmacro{\End}{#2+1} 
% \draw[#3,shorten >=2pt]
% \foreach \i in {#1,...,\End}{%
%    (\i,0) to  (\i-1,0)
%} ; 
%\node[color=Red] at (#2,-0.75) {\small End };
%\node[color=Green] at (#1,-0.75) {\small Start };
% \pgfmathsetmacro{\xtxt}{(#1+#2)/2}     
%\node at (\xtxt,0.5) {\small Hop \number\numexpr-#2+#1\relax\ units to the \emph{left}};
%\end{tikzpicture}} 







%%%%%%%%%%%%%%%%%%%%%%%%%%%%%%%%%%%%%%%%%%%%%%%%%%%%%%%%%%%%%%%%%%%%%%%%%%%
%%%%%%%%%%%%%%%%%%%%%%%%%%%%%%%%%%%%%%%%%%%%%%%%%%%%%%%%%%%%%%%%%%%%%%%%%%%
%%%%%%%%%%%%%%%%%%%%%%%%%%%%%%%%%%%%%%%%%%%%%%%%%%%%%%%%%%%%%%%%%%%%%%%%%%%
%%%%   Grayscale versions of highlight, boxes, examples, etc.          %%%%
%%%%%%%%%%%%%%%%%%%%%%%%%%%%%%%%%%%%%%%%%%%%%%%%%%%%%%%%%%%%%%%%%%%%%%%%%%%
%%%%%%%%%%%%%%%%%%%%%%%%%%%%%%%%%%%%%%%%%%%%%%%%%%%%%%%%%%%%%%%%%%%%%%%%%%%
%%%%%%%%%%%%%%%%%%%%%%%%%%%%%%%%%%%%%%%%%%%%%%%%%%%%%%%%%%%%%%%%%%%%%%%%%%%

% grayscale highlight

\newcommand{\hl}[1]{\colorbox{gray!40}{#1}}




\newcommand{\mynumberline}[3]{%
\begin{tikzpicture}[out=45,in=135,relative,>=stealth]
\draw[<->] (#1-1,0)--(#2+1,0);
\foreach \x in {\number\numexpr#1\relax,...,\number\numexpr#2\relax}  
\draw[shift={(\x,0)},color=black] (0pt,2pt) -- (0pt,-2pt) node[below] {\footnotesize $\x$};

\pgfmathsetmacro{\End}{#2-1} 
 \foreach \i in {#1,...,\End}{%
    (\i,0) to (\i+1,0)
} ; 
\end{tikzpicture}}

\newcommand{\mynumberray}[3]{%
\begin{tikzpicture}[out=45,in=135,relative,>=stealth]
\draw[->] (#1,0)--(#2+1,0);
\foreach \x in {\number\numexpr#1\relax,...,\number\numexpr#2\relax}  
\draw[shift={(\x,0)},color=black] (0pt,2pt) -- (0pt,-2pt) node[below] {\footnotesize $\x$};

\pgfmathsetmacro{\End}{#2-1} 
 \foreach \i in {#1,...,\End}{%
    (\i,0) to (\i+1,0)
} ; 
\end{tikzpicture}}

\newcommand{\mynumberlinewithpoint}[3]{%
\begin{tikzpicture}[out=45,in=135,relative,>=stealth]
\draw[<->] (#1-1,0)--(#2+1,0);
\foreach \x in {#3}
\fill [color=black] (\x,0) circle (2.25pt);
\foreach \x in {\number\numexpr#1\relax,...,\number\numexpr#2\relax}  
\draw[shift={(\x,0)},color=black] (0pt,2pt) -- (0pt,-2pt) node[below] {\footnotesize $\x$};

\pgfmathsetmacro{\End}{#2-1} 
 \foreach \i in {#1,...,\End}{%
    (\i,0) to (\i+1,0)
} ; 
\end{tikzpicture}}






\newcommand{\mysquarecoordplane}[2]{%
\begin{tikzpicture}[scale=.5, out=45,in=135,relative,>=stealth]
\draw[<->] (#1-1,0)--(#2+1,0);
\foreach \x in {\number\numexpr#1\relax,...,-1}  
\draw[shift={(\x,0)},color=black] (0pt,2pt) -- (0pt,-2pt) node[below] {\tiny $\x$};

\foreach \x in {1,...,\number\numexpr#2\relax}  
\draw[shift={(\x,0)},color=black] (0pt,2pt) -- (0pt,-2pt) node[below] {\tiny $\x$};

\pgfmathsetmacro{\End}{#2-1} 
 \foreach \i in {#1,...,\End}{%
    (\i,0) to (\i+1,0)
} ; 

\draw[<->] (0,#1-1)--(0,#2+1);
\foreach \x in {\number\numexpr#1\relax,...,-1}  
\draw[shift={(0,\x)},color=black] (2pt,0pt) -- (-2pt,0pt) node[right] {\tiny $\x$};
\foreach \x in {1,...,\number\numexpr#2\relax}  
\draw[shift={(0,\x)},color=black] (2pt,0pt) -- (-2pt,0pt) node[right] {\tiny $\x$};
\pgfmathsetmacro{\End}{#2-1} 
 \foreach \i in {#1,...,\End}{%
    (0,\i) to (0,\i+1)
} ;

\end{tikzpicture}}






\newcommand{\addsubnumlinetoright}[3]{%
\begin{tikzpicture}[out=45,in=135,relative,>=stealth]
\draw[<->] (#1-2,0)--(#2+2,0);
\foreach \x in {\number\numexpr#1-1\relax,...,\number\numexpr#2+1\relax}  
\draw[shift={(\x,0)},color=black] (0pt,2pt) -- (0pt,-2pt) node[below] {\footnotesize $\x$};
\fill (#1,0) circle (2pt);
\fill (#2,0) circle (2pt);

\pgfmathsetmacro{\End}{#2-1} 
 \draw[#3,shorten >=2pt,color=black]
 \foreach \i in {#1,...,\End}{%
    (\i,0) to (\i+1,0)
} ; 
\node[color=black] at (#2,-0.75) {\small End};
\node[color=black] at (#1,-0.75) {\small Start};
 \pgfmathsetmacro{\xtxt}{(#1+#2)/2}
\node at (\xtxt,0.5) {\small Hop \number\numexpr#2-#1\relax\ units to the \emph{right}};
\end{tikzpicture}}




\newcommand{\symmnumbline}[3]{%
\begin{tikzpicture}[out=45,in=135,relative,>=stealth]
\draw[<->] (#1-2,0)--(#2+2,0);
\foreach \x in {\number\numexpr#1-1\relax,...,\number\numexpr#2+1\relax}  
\draw[shift={(\x,0)},color=black] (0pt,2pt) -- (0pt,-2pt) node[below] {\footnotesize $\x$};
%\fill (#1,0) circle (2pt);
%\fill (#2,0) circle (2pt);

\pgfmathsetmacro{\Start}{#1-1} 
 \draw[#3,shorten >=2pt,color=black]
 \foreach \i in {\Start,...,-1}{%
    (\i,0) to (-\i,0)
} ; 

\end{tikzpicture}}



\newcommand{\disttozeronl}[3]{%
\begin{tikzpicture}[out=45,in=135,relative,>=stealth]
\draw[<->] (-3,0)--(3,0);
\foreach \x in {\number\numexpr-3\relax,...,\number\numexpr3\relax}  
\draw[shift={(\x,0)},color=black] (0pt,2pt) -- (0pt,-2pt) node[below] {\footnotesize $\x$};
%\fill (1,0) circle (2pt);
%\fill (0,0) circle (2pt);

\pgfmathsetmacro{\End}{#1+#2-1} 
 \draw[#3,shorten >=0pt,color=black]
 \foreach \i in {1}{%
    (\i-1,0) to (\i,0)
} ; 
\end{tikzpicture}}


\newcommand{\myaddsubnumlinetoright}[3]{%
\begin{tikzpicture}[out=45,in=135,relative,>=stealth]
\draw[<->] (#1-2,0)--(#1+#2+2,0);
\foreach \x in {\number\numexpr#1-1\relax,...,\number\numexpr#1+#2+1\relax}  
\draw[shift={(\x,0)},color=black] (0pt,2pt) -- (0pt,-2pt) node[below] {\footnotesize $\x$};
\fill (#1,0) circle (2pt);
\fill (#1+#2,0) circle (2pt);

\pgfmathsetmacro{\End}{#1+#2-1} 
 \draw[#3,shorten >=2pt,color=black]
 \foreach \i in {#1,...,\End}{%
    (\i,0) to (\i+1,0)
} ; 
\node[color=black] at (#1+#2,-0.75) {\small End};
\node[color=black] at (#1,-0.75) {\small Start};
 \pgfmathsetmacro{\xtxt}{(#1+#1+#2)/2}
\node at (\xtxt,0.5) {\small Hop  \number\numexpr#2\relax\ units to the \emph{right}};
\end{tikzpicture}}


\newcommand{\addsubnumlinetoleft}[3]{%
\begin{tikzpicture}[out=135,in=45,>=stealth]
\draw[<->] (#2-2,0)--(#1+2,0);
\foreach \x in {\number\numexpr#2-1\relax,...,\number\numexpr#1+1\relax}
\draw[shift={(\x,0)},color=black] (0pt,2pt) -- (0pt,-2pt) node[below] {\footnotesize $\x$};
\fill (#1,0) circle (2pt);
\fill (#2,0) circle (2pt);

\pgfmathsetmacro{\End}{#2+1} 
 \draw[#3,shorten >=2pt,color=black]
 \foreach \i in {#1,...,\End}{%
    (\i,0) to  (\i-1,0)
} ; 
\node[color=black] at (#2,-0.75) {\small End };
\node[color=black] at (#1,-0.75) {\small Start };
 \pgfmathsetmacro{\xtxt}{(#1+#2)/2}     
\node at (\xtxt,0.5) {\small Hop \number\numexpr-#2+#1\relax\ units to the \emph{left}};
\end{tikzpicture}} 















%%%%%%%%%%
% blue box with title bold and in-line
%%%%%%%%%%


\newenvironment{bluebox}[1][]{%
\begin{mdframed}[%
	skipabove=\baselineskip plus 2pt minus 1pt,
    skipbelow=\baselineskip plus 2pt minus 1pt,
innertopmargin=4pt,
innerleftmargin=4pt,
innerrightmargin=4pt,
linewidth=0.0pt,
linecolor=halfgray,
backgroundcolor=halfgray,
userdefinedwidth=10.5cm,
align=center,
roundcorner=4pt
]\noindent\textbf{#1}\qquad%
}{%
    \end{mdframed}
}



%%%%%%%%%%
% blue box with title bold and separate line
%%%%%%%%%%


\newenvironment{titlebox}[1][]{%
    \begin{mdframed}[%
        frametitle={\large #1},
        skipabove=\baselineskip plus 2pt minus 1pt,
        skipbelow=\baselineskip plus 2pt minus 1pt,
        linewidth=0.5pt,
linecolor=halfgray,
        frametitlebackgroundcolor=halfgray,
backgroundcolor=halfgray,
userdefinedwidth=10.5cm,
align=center,
roundcorner=4pt
    ]%
}{%
    \end{mdframed}
}


%%%%%%%%%%
% wide blue box with title bold and separate line
%%%%%%%%%%


\newenvironment{widetitlebox}[1][]{%
    \begin{mdframed}[%
        frametitle={\large #1},
        skipabove=\baselineskip plus 2pt minus 1pt,
        skipbelow=\baselineskip plus 2pt minus 1pt,
        linewidth=0.5pt,
linecolor=halfgray,
        frametitlebackgroundcolor=halfgray,
backgroundcolor=halfgray,
userdefinedwidth=14.5cm,
align=center,
roundcorner=4pt
    ]%
}{%
    \end{mdframed}
}


%%%%%%%%%%
% Exercises Heading
%%%%%%%%%%

\newcommand{\exercises}{\newpage{\color{halfgray}\exertitle \noindent Exercises\,\,}{\exercs\color{halfgray}\arabic{chapter}.\arabic{section}} {\color{halfgray}\leaders\hrule height .8ex depth \dimexpr-.8ex+0.8pt\relax\hfill
\vspace{3mm}} }

%%%%%%%%%%
% fancy example style
%%%%%%%%%%


\theoremstyle{definition}
\newtheorem{exinn}{Example}[section]
\newenvironment{example}
  {\clubpenalty=10000
   \begin{exinn}%
   \mbox{}%
   {\color{halfgray}\leaders\hrule height .8ex depth \dimexpr-.8ex+0.8pt\relax\hfill}%
   \mbox{}\linebreak\ignorespaces}
  {\par\kern2ex\color{halfgray}\hrule\end{exinn}}



%%%%%%%%%%
% On-Your-Own example 
%%%%%%%%%%

\newtheoremstyle{noperiod}% name
  {.5cm}%      Space above
  {.5cm}%      Space below
  {}%         Body font
  {}%         Indent amount (empty = no indent, \parindent = para indent)
  {\bfseries}% Thm head font
  {}%        Punctuation after thm head
  {.5em}%     Space after thm head: " " = normal interword space;
        %       \newline = linebreak
  {}%         Thm head spec (can be left empty, meaning `normal')

\theoremstyle{noperiod}
\newtheorem*{noperiod}{On Your Own \ldots}
\newenvironment{oyoe}
  {\clubpenalty=10000
   \begin{noperiod}%
   \mbox{}%
   {\color{halfgray}\leaders\hrule height .8ex depth \dimexpr-.8ex+0.8pt\relax\hfill}%
   \mbox{}\linebreak\ignorespaces}
  {\par\kern2ex\color{halfgray}\hrule\end{noperiod}}
