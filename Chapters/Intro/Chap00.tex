\chapter{General}
\section{Administration}

\begin{itemize}\itemsep=0mm
  \item[] Head of School, \emph{Scott Schamberger}
  \item[] Assistant Head of School for Academic Affairs, \emph{Jonathan Gray PhD}
  \item[] Dean of Faculty, \emph{Weslie Wald}
  \item[] Dean of Students, \emph{Hunter Wolfe}
  \item[] Director of College Advising, \emph{Amelia Johnson}
\end{itemize}

\section{Departments}

\begin{itemize}\itemsep=0mm
  \item[] Arts, \emph{Clay Colvin, Chair}
  \item[] Computer Science \& Engineering, \emph{William Belser '80, Chair}
  \item[] English, \emph{James Griffin, Chair}
  \item[] History, \emph{Kelly Jacobs, Chair}
  \item[] Languages, \emph{William Blackerby  '05, Chair}
  \item[] Mathematics, \emph{Chris Mullinax, Chair}
  \item[] Physical Education, \emph{Brad Skiff, Chair}
  \item[] Science, \emph{Tessa Magnuson, Chair               }
\end{itemize}

\section{Committees with Academic Responsibilities}
\begin{itemize}\itemsep=0mm
  \item[] Academics Committee, \emph{Jonathan Gray and Weslie Wald, Chairs}
\bmc{2}\begin{itemize}\itemsep=0mm
  \item[] Clay Colvin
  \item[] William Belser
  \item[] James Griffin
  \item[] Kelly Jacobs
  \item[] William Blackerby
  \item[] Chris Mullinax 
  \item[] Brad Skiff 
  \item[] Tessa Magnuson    
  \item[] Amelia Johnson             
  \item[] Jourdan Cunningham
    \item[] Commissioners of Education
\end{itemize}\etc
  \item[] Student External Engagement Committee, \emph{Chris Tetzlaff and Hazal Mohammed, Chairs}
\end{itemize}





\section{Faculty}
\bmc{2}
\begin{itemize}\itemsep=1mm
\item[]D'Anthony Allen, English
\item[]Neil Barrett, English
\item[]Jean Bassene, Languages
\item[]William Belser, Computer Science \& Engineering
\item[]William Blackerby, Languages
\item[]John Brunzell, Mathematics
\item[]Athena Chang, Languages
\item[]Renee Chow PhD, English
\item[]Dan Clinkman PhD, History
\item[]Clay Colvin, Arts
\item[]Bob Cooper PhD, History
\item[]Colin Davis PhD, History
\item[]Emanual Ellinas, Arts
\item[]Jim Flaniken, Mathematics
\item[]Jonathan Gray PhD, Mathematics
\item[]James Griffin, English
\item[]Jonathan Horn PhD, Languages
\item[]Leslie Hurt, Science
\item[]Kelly Jacobs, History
\item[]Hye Sook Jung PhD, Arts
\item[]Mac Lacasse PhD, Mathematics
\item[]Tessa Magnuson, Science
\item[]George Mange, Languages
\item[]Pedro Mayor, Languages
\item[]Hazal Mohammed, Science
\item[]Chris Mullinax, Mathematics
\item[]Rebecca Neel, Mathematics
\item[]Dane Peterson, Arts
\item[]Justin Pino, Physical Education
\item[]Michael Sheehan, Arts
\item[]Jeffrey Sides PhD, Science
\item[]Brad Skiff, Physical Education
\item[]Chris Tetzlaff, Science
\item[]Stephanie Thomas, Mathematics
\item[]Greg Van Horn, Physical Education
\item[]Lauren Wainwright JD, History
\item[]Weslie Wald, Languages
\item[]Hunter Wolfe, History
\item[]Cal Woodruff, English
\item[]Lee Wright PhD, Arts
\end{itemize}
\etc


\section{Graduation Requirements}

\begin{tabular}{lll}
  Department & Credits & Comments\\
  \hline  \hline\\
  Arts & 1 credit& 0.5 credits in Art History or Music History\\
   & & 0.5 credits in Arts\\  
   \\
  English &  4 credits& At least one credit per year in grades 9-11  \\
  \\
  History      &  3 credits&  1 credit of World History:  To 1500\\
              &  &  1 credit of AP World History or AP European History\\  
              &  &  1 credit of AP United States History\\       
   \\                              
  Languages  &  3 credits& Must be in same language  \\
  \\
  Mathematics   &  3 credits& Must include 1 credit at Algebra II level or higher  \\
  \\
  Physical Education   &  3 credits& 0.5 credits WellFit and 0.5 credits 9th grade PE\\
&&  1.0 credit in each of 10th grade PE and 11th grade PE  \\
\\
  Science                    &  3 credits & Must complete 1 credit in each of Biology, Chemistry, and Physics \\
  \\
  Any                        &  3 credits\\
\end{tabular}


\section{Course Enrollment Requirements}

In general, students are required to enroll in seven, six, and five (Grades 8, 9-11, and 12, resp.) courses per semester.\footnote{10th and 11th Grade PE are not used in enrollment counts.}  At least four core subjects (English, History, Languages, Mathematics, Science) must be represented each semester; an MSON course or Independent Study cannot be used to reach the minimum course enrollment for a semester and will necessarily be the seventh (11th grade) or sixth (12th grade) course.  Any deviation from the indicated enrollments must be approved by the Assistant Head of School for Academic Affairs.

To enroll in seven or more courses in grades 9-12, an Academic Overload form must be submitted to the Academics Committee for approval.  Similarly, if a student wishes to enroll in two or more courses in a core subject, the corresponding form must be submitted to the Academics Committee for approval. 

\begin{itemize}\itemsep=0mm
  \item[] \textbf{Grade 8}
  
  Students in 8th grade are required to enroll in 
\begin{enumerate}\itemsep=0mm
    \item Art 8
    \item English 8
    \item 8th Grade Social Studies
    \item A Chinese, French, Latin, or Spanish course 
    \item A mathematics course
    \item PE 8
    \item Science 8
  \end{enumerate}
  

  
  \item[] \textbf{Grade 9}
  
  Students in 9th grade are required to enroll in 
\begin{enumerate}\itemsep=0mm
    \item English 9
    \item World History:  To 1500
    \item A Chinese, French, Latin, or Spanish course 
    \item A mathematics course
    \item WellFit and PE 9
    \item Biology
  \end{enumerate}
  
  \item[] \textbf{Grade 10}
  
  Students in 10th grade are required to enroll in 
\begin{enumerate}\itemsep=0mm
    \item Critical Reading \& Analytical Writing
    \item AP World History or AP European History
    \item A Chinese, French, Latin, or Spanish course 
    \item A mathematics course
    \item Art History or Music History
    \item Chemistry
    \item 10th Grade PE
  \end{enumerate}
  
  An additional semester elective must be chosen to complement Art History or Music History thereby bringing the total course enrollments to six per semester (not including PE).
  
  \item[] \textbf{Grade 11}
  
  Students in 11th grade are required to enroll in 
\begin{enumerate}\itemsep=0mm
    \item AP English Language or Two English Electives
    \item AP United States History
    \item A Chinese, French, Latin, or Spanish course 
    \item A mathematics course
    \item 11th Grade PE
  \end{enumerate}
  
  Additional courses must be chosen to bring the total course enrollments to six courses per semester (not including PE).
  
  \item[] \textbf{Grade 12    }
  
  Students in 12th grade are required to enroll in AP English Language or Two English Electives.  Additional courses must be chosen to bring the total course enrollments to five courses and at least four core subjects (English, History, Languages, Mathematics, Science) are represented each semester.

\end{itemize}

\section{Grading Scale and GPA}

A student’s grade point average (GPA) is calculated at the end of each year to reflect our cumulative grading model. Year and cumulative GPAs are recorded on the transcript each year.  Independent Studies, MSON courses, 10th Grade PE, and 11th Grade PE are not included in GPA calculations. \footnote{For calculation purposes, these courses have $0.0$ quality points possible.}

Starting in the Class of $2024$, the GPA calculation was changed to an unweighted $4.0$ system wherein the quality points earned are jointly proportional to the numerical grade earned in the course and the grade point credits for the course. E.g., if a student earns a grade of $87$ in a $1.0$ course, then the quality points earned are $0.87\cdot 4.0\cdot 1.0 = 3.48$.  A more comprehensive example follows:
\begin{center}
\begin{tabular}{|l||l|l|l|}
  Course &Grade& Quality Points Possible& Quality Points Earned\\
  \hline
  \hline
 English 9      &   $91  $  & $1.0$ &  $0.91= 0.91\cdot 1.0$\\ 
 \hline                                
 World History:  To 1500  &   $ 86 $  &  $1.0$ &  $0.86 =0.86\cdot 1.0$\\ 
 \hline                                 
 Latin II       &   $94  $  &  $1.0$ &  $ 0.94=0.94\cdot 1.0$\\ 
 \hline                                
 Adv Geometry    &   $82  $  &  $1.0$ &  $0.82= 0.82\cdot 1.0$\\ 
 \hline                                
 WellFit        &   $90  $  &  $0.5$ &  $0.45= 0.90\cdot 0.5$\\ 
 \hline                                 
 PE 9         &   $ 100$  &  $0.0$  &  $0.00= 1.00\cdot 0.0$\\ 
 \hline                                 
 Biology     &   $ 78 $  &  $1.0$ &  $0.78= 0.78\cdot 1.0$\\ 
 \hline 
 \hline
 Sum Total &&$5.5$ & $4.76$ \\
 \hline
\end{tabular}

\end{center}

The GPA earned for this year would then be $(4.76/5.5)\cdot 4.0 = 3.46.$   In general, let $p_1,p_2,\ldots, p_k$ be the quality points possible for the respective courses wherein a particular student earned  grades $g_1,g_2,\ldots, g_k.$  The GPA corresponding to these $k$ courses can be calculated by\footnote{Equivalently, one can take the dot product of the $Q$ and $G$ vectors, divide  the latter result  by $Q$ in $\ell_1$ norm, and then multiply by $4.$}

$$\displaystyle GPA = 4.0\times \frac{\displaystyle\sum_{i=1}^k g_i\cdot q_i}{\displaystyle\sum_{i=1}^k q_i}$$








Note:  While not reflected on transcripts, faculty may use the following grade translation table between numerical and letter grades:


	$$\underbrace{ \hspace{5mm} 97}_{A+}
  \underbrace{\hspace{5mm} 93}_{A}
  \underbrace{\hspace{5mm} 90}_{A-}
  \underbrace{\hspace{5mm} 87}_{B+}
  \underbrace{\hspace{5mm} 83}_{B}
  \underbrace{\hspace{5mm} 80}_{B-}
  \underbrace{\hspace{5mm} 77}_{C+}
  \underbrace{\hspace{5mm} 73}_{C}
  \underbrace{\hspace{5mm} 70}_{C-}
  \underbrace{\hspace{5mm} 67}_{D+}
  \underbrace{\hspace{5mm} 63}_{D}
  \underbrace{\hspace{5mm} 60}_{D-}
  \underbrace{\hspace{5mm} 0}_{F}$$
  
  


%\begin{tabular}{ccccccccccccc}
%  A+ & A & A- & B+ & B & B- & C+ & C & C- & D+ & D & D- & F\\ 
%\end{tabular}

\section{Academic Policies}

\begin{enumerate}
  
  \item \emph{Drop/Add}
  
  
  \item \emph{External Coursework Reflected in GPA} 
  
  GPAs recorded on the transcript include only coursework completed at Indian Springs.  Because grading scales and course requirements vary from school to school, we do not print courses taken at other schools on our transcript nor do we include them in the GPA.  When the student applies to college, any transcripts from other schools recording grades from 9th grade and above are sent alongside the Indian Springs transcript.

  \item \emph{Academic Overload }
  
  Students in grades 9-12 who wish to enroll in seven or more courses in a semester must complete the appropriate form. Students must obtain the signature of their parent and advisor.  The form is then provided to the Academics Committee for review. The Assistant Head of School for Academic Affairs will evaluate performance of the student during the first quarter of the school year. Students who are struggling in their overload class will be asked to remove a course at that time.
  
  \item \emph{Departmental Overload}
  
  Students who wish to enroll in more than one course in a department during a semester must complete the appropriate form.  Students must obtain the signature of their parent, advisor, and department chair.  The form is then provided to the Academics Committee for review.  The Assistant Head of School for Academic Affairs will review performance of the student during the first quarter of the school year. Students who are struggling in one or both classes will be asked to remove a course at that time.
  
  \item \emph{AP Exam Requirements}
  
  The school deadline  for choosing to take an  AP exam for a course in which they \underline{are enrolled} is the last day of Fall classes.  This deadline is after the CollegeBoard's deadline.  The CollegeBoard does not charge a fee if a student registers for an AP exam and cancels prior to their published date (typically mid-November).  If a student cancels after that date, the CollegeBoard applies an ``unused/canceled exam fee'' per unused/canceled exam.  
  
  If a student wishes to take an AP exam for a course in which  they \underline{are not enrolled}:
  
  \begin{enumerate}
    \item Complete the required form and submit it to the Assistant Head of School for Academic Affairs by the second Friday in September. The form may be submitted on paper or by email.
    \item The form will be reviewed and an approval will be considered based upon academic standing, exam preparation, exam load, scheduling constraints, and faculty interviews. The decision will be communicated by email.
    \item  If approved, you must contact the AP Coordinator by the last day of classes in September to confirm your intent to take the exam. The standard cost per AP exam will be billed home once confirmed. 
  \end{enumerate}






  

\end{enumerate}





