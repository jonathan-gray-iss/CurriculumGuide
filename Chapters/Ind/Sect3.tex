\section{The Capstone Program}


\subsection{General Description}

The Indian Springs School Capstone is a program wherein students may explore an area of interest that extends beyond and augments the Springs classroom experience and Independent Study Program.  Students will be active participants in the learning process and will be the protagonist in their learning.  A student-chosen faculty mentor will assist the student through the process by offering support, providing guidance, forming community connections, procuring necessary materials, and so on.  The experience is split into three phases:  Process, Progress, and Product.

A successful Capstone marries the abstract with the practical, knowledge with life.  It bridges disciplines by being interdisciplinary and/or transdisciplinary.  It extends a student’s learning and, among its best forms, brings in or gives back to the broader community as part of the learning.



In the Process Phase, 11th grade students are introduced to the program in a class meeting at the start of the spring semester.  Interested students use the next month to research potential capstone ideas.  A final capstone idea is turned into a proposal, which will be reviewed by the Capstone Review Committee.\footnote{Sra. Wald, Mr. Wolfe, Dr. Gray}  The results of the review will be provided to the student within two weeks of the proposal submission due date.  The student will use the following two months to review their capstone project in a more thorough manner.  This review will include a general timeline including tangible outcomes to be provided at three forthcoming check-ins.  Also, depending on the nature of the capstone, the student may provide a literature review, perform strategic readings, procure necessary resources, consult with experts, site-plan, secure approval by necessary organizations, and so on.  In other words, the plan for how all foreseeable aspects of the capstone will be created and then provided to the Capstone Review Committee for approval.  If revisions, clarifications, etc. are needed, the student will have until the end of the spring semester to resubmit their Capstone Plan.

The Progress Phase should take place over the course of the summer between the student’s 11th and 12th grade years.  The student and their mentor will meet regularly for informal check-ins, usually for an hour each week.  Two formal check-ins will take place during June and July where the student will grade themself on meeting the goals provided in their Plan.  A final, formal check-in will be held as a group in August to share the mutual progress made by all.  It is suggested that all capstone students meet over the summer as a learning cohort in addition to the group meeting in August.

For the Product Phase, every Capstone will culminate in a presentation of the artifacts of the student’s learnings.  A written document will be among these artifacts and may take the form of a general thesis; an expository paper; a formal analysis, explanatory work, or creation statement as it relates to artistic work, and so on.  

As part of their Capstone Proposal, the student indicates a Capstone Defense Committee.  The Committee should include the faculty mentor, a member of a Humanities department, and a member of a STEM department.  These three members will deliberate on the learnings conveyed from the capstone and assign a final grade.





























\subsection{Capstone Proposal Procedure}

Provide or address the following in essay\footnote{ This should be a google doc shared with Dr. Gray.} form:
\begin{enumerate}\itemsep=0mm
\item Name
\item Mentor name
\item Humanities\footnote{Humanities:  Arts, English, History, Languages
} and STEM\footnote{STEM:  Computer Science \& Engineering, Math, Science} Committee members
\item Capstone topic or area
\item Questions you wish to ask and/or learnings you wish to gain
\item Why the capstone topic is meaningful to you
\item Prior experience in proposed capstone area\footnote{``In a capstone course, students synthesize, integrate, and/or apply their previous knowledge, rather than acquire new knowledge or skills. Students demonstrate mastery, not learn new knowledge/skills.'' [\href{https://registrar.oregonstate.edu/sites/registrar.oregonstate.edu/files/additional_resources_capstone.pdf}{Source}]}
\item Role your mentor will play in the process
\item Not including your mentor, who will assist you with accountability
\item Materials and/or travel necessary along with a tentative budget\footnote{Approval of a Capstone does not imply school funding for the Capstone
}
\item Challenges you foresee in completing the capstone
\item Artifacts you expect to produce as evidence of successful completion
\item Expected form of written document
\item General structure of your committee presentation

\end{enumerate}

\subsection{Capstone Plan Requirements}

In whatever form that is most reasonable and organized, provide and/or address the following:

\begin{enumerate}\itemsep=0mm

\item A timeline you will follow showing the phases of your capstone and self-identified checkpoints
\item How you will measure your progress at the first and second check-ins
\item Literature review
\item Bibliography in a style appropriate for the discipline
\item Materials and/or travel along with a firm budget
\item Experts you may need to consult and in what capacity those consultations are needed
\item An analysis of any sites/travel that are necessary to carry out the capstone
\item Progress made in securing approval for any permitted/authorized aspects of the capstone

\end{enumerate}

\subsection{Capstone Timeline}

 \textbf{Process Phase}\\

\begin{tabular}{ll}
  Jan 16&	Presentation to Rising Seniors\\
  Feb 23&	Capstone Proposals Due\\
  Mar 8	&Notice of Approved Capstones\\
  May 3	&Capstone Plan Due\\
\end{tabular}\\


\noindent \textbf{Progress Phase}\\

\begin{tabular}{ll}
  Jun 21	&	First Progress Check-In\\
  Jul 19	&	Second Progress Check-In\\
  Aug TBD	&	Capstone Group Meeting Check-In (All Participants)\\
\end{tabular}\\



\noindent \textbf{Product Phase  }\\

\begin{tabular}{ll}
 Sep 20	&	Capstone Artifacts Due\\
 Oct 1-4	&	Presentations 
\end{tabular}


