\section{Current Courses}

\noindent\textbf{Art 8} \hfill Various

\noindent Fall and Spring Semesters - 0.5 Credits

\vspace{1mm}\emph{The 8th grade art wheel allows each student to work with the arts faculty to explore the ways the arts can impact their general experience at Springs and beyond. Students are led through projects that help them become better creative and critical thinking through questions like: What is the difference between Hearing and Listening? How does music impact your experience? How can concepts like measurement and precision help us design and communicate/express ideas? How can we use drawing to record and communicate? How can digital tools help express ideas? How is photography a way to appreciate memory? How is photography a model for appreciating technical crafts and the care that goes into the process of making a photograph? How is singing a way to express oneself individually or as a community? How does your community connect through song? (Or shared stories?) How do writers and stage designers literally set the stage to tell stories? How is communication an important part of a community? How can we be better communicators and storytellers?}\\


\noindent\textbf{Drawing and Design} \hfill Colvin

\noindent Fall and Spring Semesters - 0.5 Credits

\vspace{1mm}\emph{In this course, students explore the elements of art and principles of design as well as various approaches to drawing and painting. They discover how to create space and form through mark making, value, perspective, and color.}\\


\noindent\textbf{Illustration} \hfill Colvin

\noindent Fall and Spring Semesters - 0.5 Credits

\vspace{1mm}\emph{The Illustration course has been described by students as “a book club that is also a drawing club”. It encourages students to engage with text in a new way, and become deeper readers. Texts used are intentionally varied, but there is focus on Ray Bradbury’s “The Illustrated Man” and Italo Calvino’s series of stories “Marcovaldo” and “Invisible Cities”. Good Reads website says of Marcovaldo: “Marcovaldo is an unskilled worker in a drab industrial city in northern Italy. He is an irrepressible dreamer and an inveterate schemer. Much to the puzzlement of his wife, his children, his boss, and his neighbors, he chases his dreams - but the results are never the ones he had expected.”}\\


\noindent\textbf{Sculpture} \hfill Colvin

\noindent Fall and Spring Semesters - 0.5 Credits

\vspace{1mm}\emph{In this course, students explore the elements of art and principles of 3D design as well as various techniques of working in ceramics.}\\


\noindent\textbf{Art History} \hfill Colvin

\noindent Fall Semester - 0.5 Credits

\vspace{1mm}\emph{Art History is one of the broadest and deepest disciplines in the humanities. In the 10th grade semester survey course students will examine the visual arts from the Paleolithic era to the present. The course will employ a variety of critical, theoretical and methodological perspectives and approaches. The main goal is to equip students with visual literacy to allow them to effectively navigate the contemporary experience}\\


\noindent\textbf{Advanced Methods in Drawing} \hfill Colvin

\noindent Spring Semester - 0.5 Credits

\vspace{1mm}\emph{This class builds on the skills and concepts introduced in the beginning level class. Students gain an understanding of the qualities of a wider range of media, choosing the appropriate material for the desired form of expression. Initially, the goal is to strengthen representational skills. Later projects demand greater expressiveness or inventiveness. Students gain an artistic vocabulary and experience in analyzing works of art, by both master artists and each other.}\\


\noindent\textbf{Advanced Methods in Sculpture} \hfill Colvin

\noindent Spring Semester - 0.5 Credits

\vspace{1mm}\emph{This class continues exploring both functional and non-functional three-dimensional design. Students are asked to find various means of organizing and interpreting form, making creative thinking as important as technique. There is greater individual choice of materials within the format of projects involving elements and principles of design.}\\


\noindent\textbf{AP Studio Art} \hfill Colvin

\noindent Spring Semester - 0.5 Credits

\vspace{1mm}\emph{Students in the AP Visual Art courses work on their line of inquiry either within the College Board requirements or in a more personalized structure. The course encourages research, experimentation, and revision. Each student presents their work either on campus or in regional and state competitions. They continue to increase their levels of understanding and skills. Each student should be comfortable signing in, working in, and sharing their work via Adobe platforms or Google platforms. Prerequisite: Instructor Approval.}\\


\noindent\textbf{Digital Photography} \hfill TBD

\noindent Fall and Spring Semesters - 0.5 Credits

\vspace{1mm}\emph{Students in this class learn how to create their own Digital photography from beginning to end. Instruction includes the capturing of quality images, image editing, and printing.}\\


\noindent\textbf{Introduction to Black \& White Photography} \hfill TBD

\noindent Fall and Spring Semesters - 0.5 Credits

\vspace{1mm}\emph{Students in this class learn how to create their own photography from beginning to end. Instruction includes the secrets of capturing quality images on film, development, and custom printing. Field trips are included. Students exhibit their work prior to the end of the term. A 35mm SLR camera with manual settings is required.}\\


\noindent\textbf{Advanced Methods in Photography} \hfill TBD

\noindent Fall and Spring Semesters - 0.5 Credits

\vspace{1mm}\emph{The Advanced Photography course reinforces the value of the knowledge and techniques students have learned in previous experiences with photography.  This course encourages more conceptual approaches to prompts.}\\


\noindent\textbf{Adobe Photoshop I} \hfill TBD

\noindent Fall and Spring Semesters - 0.5 Credits

\vspace{1mm}\emph{This course will cover all basic and some advanced techniques in Adobe Photoshop. Students will produce a number of images through a series of projects incorporating their original photography.}\\


\noindent\textbf{Yearbook Design and Layout} \hfill TBD

\noindent Spring Semester - 0.5 Credits

\vspace{1mm}\emph{The Year Book Design class builds skills within a number of digital platforms, as well as gaining deeper understanding of the elements of art and principles of design. The students directly design and create the year book for the school. Through this activity they explore questions like: How is photography a way to appreciate memory? How is photography a model for appreciating technical crafts and the care that goes into the process of making a photograph? How does your community connect through images? (Or shared stories?)}\\


\noindent\textbf{Acting I} \hfill Peterson

\noindent Spring Semester - 0.5 Credits

\vspace{1mm}\emph{Open to all students grade 9-12, whether novice actor or veteran performer, this course introduces the basics of acting: concentration, relaxation, observation, and characterization. Students learn juggling and pantomime techniques and use daily improvisations and theater games to build characterization skills and create original material for performance. At the end of the semester, they perform short scenes and monologues, both scripted and original. Additionally, students attend and critique local theater performances.}\\


\noindent\textbf{Advanced Performance Ensemble} \hfill Peterson/Wright

\noindent Spring Semester - 0.5 Credits

\vspace{1mm}\emph{This course is based on the notion that ensemble performance is an artistic and aesthetic experience that emphasizes the “whole” (we), rather than the “self” (me). Through a process-oriented collaboration, we will explore aspects of drama, music, voice, movement, and improvisational skills. Participants will create various performance demonstrations and a final new/original work. Along with these performances, assessments will include self-reflections, critical reading and responses, and a portfolio of in-progress and completed work. Prerequisite: Instructor Approlval.}\\


\noindent\textbf{Directing and Stage Management} \hfill Peterson

\noindent Fall Semester - 0.5 Credits

\vspace{1mm}\emph{A 40+ years Springs Tradition, this course is a senior elective that explores the theories and practices of directing and managing a theatrical production. Through readings, discussions, and rehearsals, students will choose a One-Act Play to produce on the Badhame Theater stage. Communication, problem-solving, storytelling, and collaboration are just some of the skills developed over the semester. While previous experience in theatre is helpful, it is not required.}\\


\noindent\textbf{Theatrical Design \& Stagecraft} \hfill Peterson

\noindent Fall and Spring Semesters - 0.5 Credits

\vspace{1mm}\emph{In this course, students learn the basics of set construction and scenery and lighting design. They build, paint, light a set and serve as stage crew for one major Indian Springs theater production during the semester. In addition, students set up and run lights, sound, and media for Town Meetings and other school functions on a rotating basis.}\\


\noindent\textbf{Contemporary Music Ensemble} \hfill Ellinas

\noindent Fall and Spring Semesters - 0.5 Credits

\vspace{1mm}\emph{Through a process-oriented collaboration, students will explore aspects of music, voice, sound editing, and performance. The course creates a collaborative environment where students can use and build their communication skills, as well as develop their individual musical skills and understandings. Participants will perform their songs in “the contemp concert” near the end of the semester. Along with these performances, assessments will include journaling self-reflections in response to research and experiences listening and reading.}\\


\noindent\textbf{Advanced Contemporary Music Ensemble} \hfill Ellinas

\noindent Fall and Spring Semesters - 0.5 Credits

\vspace{1mm}\emph{Through a process-oriented collaboration, students will explore aspects of music, voice, sound editing, and performance. The course creates a collaborative environment where students can use and build their communication skills, as well as develop their individual musical skills and understandings. Participants will perform their songs in “the contemp concert” near the end of the semester. Along with these performances, assessments will include journaling self-reflections in response to research and experiences listening and reading.  Auditions/Intructor permissions required.  To be held as an Evening Class.}\\


\noindent\textbf{Introduction to Music Theory} \hfill Jung

\noindent Fall Semester - 0.5 Credits

\vspace{1mm}\emph{This course introduces students to the basics of music theory. Students develop an understanding of the fundamentals of music by listening, performing, creating, and analyzing music. Topics covered include music terminology, notation skills, four-part harmonization, basic composition, music analysis, and basic ear training.}\\


\noindent\textbf{Music History} \hfill Jung

\noindent Fall Semester - 0.5 Credits

\vspace{1mm}\emph{This course examines musical style. After students develop basic skills in analysis, they apply these skills to a survey of music history. A few of the major composers, genres, forms and style characteristics are examined for each historical period. Students develop a perspective that is aural as well as verbal. All sophomores must take either this course or Tenth-Grade Art History.}\\


\noindent\textbf{Choral Conducting \& Literature} \hfill Wright

\noindent Spring Semester - 0.5 Credits

\vspace{1mm}\emph{This course introduces basic conducting techniques in a choral setting. Students will demonstrate growth and be assessed in conducting gesture, musicianship, and score study, as well as present on important musical styles throughout the history of choral music. Both Music History and Music Theory are recommended before taking this course.}\\


\noindent\textbf{AP Music Theory} \hfill Jung

\noindent Spring Semester - 0.5 Credits

\vspace{1mm}\emph{This class prepares students for the AP Music Theory exam by developing their ability to recognize, understand, and describe the basic materials and processes that are heard or presented in a score. Students develop their aural, sight-singing, written, compositional, and analytical skills through listening, performance, and analytical exercises. Topics covered include notation, keys, modes, intervals, chords, Roman numeral analysis, four-part chorale writing, musical analysis, and melodic, harmonic, and rhythmic dictation. This class is not designed for beginner-level musicians, but rather for students who are interested in the analytical aspects of music. Successful completion of the Introduction to Music Theory course is a prerequisite.}\\


\noindent\textbf{Classical Music Ensemble} \hfill Jung

\noindent Fall and Spring Semesters - 0.5 Credits

\vspace{1mm}\emph{In this yearlong course, students undertake the study and performance of selected classical works for ensembles of two or more musicians. The course consists of at least one weekly coaching session with the instructor, two required practice sessions, and one weekly musicianship class. Students enrolled in the course are required to study a minimum of four musical works and are expected to perform in at least three concerts. Students may choose to take this course for credit.}\\


\section{Past Courses}

\noindent\textbf{Advanced Art Portfolio}  - 0.5 Credits

\vspace{3mm}
\noindent\textbf{Acting II}  - 0.5 Credits

\vspace{3mm}
\noindent\textbf{Playwriting}  - 0.5 Credits

\vspace{3mm}
\noindent\textbf{Musical Theater}  - 0.5 Credits

\vspace{3mm}
\noindent\textbf{Advanced Acting}  - 0.5 Credits

\vspace{3mm}
\noindent\textbf{Musical Ensemble Performance}  - 0.5 Credits

\vspace{3mm}
\noindent\textbf{Intro to Contemporary Music Ensemble}  - 0.5 Credits

\vspace{3mm}
\noindent\textbf{Experimental Music}  - 0.5 Credits

\vspace{3mm}
\noindent\textbf{Recording Arts}  - 0.5 Credits

\vspace{3mm}
\noindent\textbf{Creating Visual Rhetoric}  - 0.5 Credits

\vspace{3mm}
\noindent\textbf{Advanced Ear Training and Harmony Application}  - 1 Credit

\vspace{3mm}
\noindent\textbf{Feminist Art}  - 0.5 Credits

\vspace{3mm}
\noindent\textbf{History of Jazz}  - 0.5 Credits

\vspace{3mm}
\noindent\textbf{Playing Shakespeare - F}  - 0.5 Credits

\vspace{3mm}
\noindent\textbf{Woodworking - F}  - 0.5 Credits

\vspace{3mm}
\noindent\textbf{Recording Arts - F}  - 0.5 Credits

\vspace{3mm}
\noindent\textbf{Exploring Abstraction}  - 0.5 Credits

\vspace{3mm}