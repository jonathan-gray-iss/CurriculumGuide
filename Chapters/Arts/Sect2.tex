\section{Course Descriptions}


\noindent\textbf{2D Design}  (\emph{Fall Semester}) \hfill Clay Colvin

{\small In this course, students explore various approaches to drawing and painting. They discover how to create space and form through mark making, value, perspective, and color. Subjects include still life, landscape, portrait, and the human figure. Learning about the principles of design strengthens students&#8217; compositions. By keeping a sketchbook/journal, students have the opportunity for further practice and exploration of individual interests.}

\vfill

\noindent\textbf{Acting I} - \emph{Fall Semester} \hfill Dane Peterson



{\small Open to all students, whether novice actor or veteran performer, this course introduces the basics of acting: concentration, relaxation, observation, and characterization. Students learn juggling and pantomime techniques and use daily improvisations and theater games to build characterization skills and create original material for performance. At the end of the semester, they perform short scenes and monologues, both scripted and original. Additionally, students attend and critique local theater performances.}

\vfill

\noindent\textbf{Art History} - \emph{Fall Semester} \hfill Clay Colvin



{\small Art History is one of the broadest and deepest disciplines in the humanities. In the 10th grade semester survey course students will examine the visual arts from the Paleolithic era to the present. The course will employ a variety of critical, theoretical and methodological perspectives and approaches.  The main goal is to equip students with visual literacy to allow them to effectively navigate the contemporary experience.}

\vfill

\noindent\textbf{Contemporary Music Ensemble} - \emph{Fall and Spring Semesters} \hfill Emanual Ellinas



{\small }

\vfill

\noindent\textbf{Digital Photography} - \emph{Fall and Spring Semesters} \hfill Michael Sheehan



{\small }

\vfill

\noindent\textbf{Intro to Black \& White Photography} - \emph{Fall and Spring Semesters} \hfill Michael Sheehan



{\small }

\vfill

\noindent\textbf{Recording Arts} - \emph{Fall and Spring Semesters} \hfill Emanual Ellinas



{\small This is a semester course designed to introduce students to the principles of sound engineering and production. Technical objectives include learning about sound and acoustics as well as developing the ``ear'' of the students; that is, increasing their ability to interpret better what they hear. Becoming familiar with equipment by hands-on experience in a multi-track recording studio is also part of the technical side. Students are exposed to analog and digital formats.


Recording Arts is largely a laboratory-type course, and therefore, attendance and participation are essential. The students take increasing control of the studio and project as the semester moves along. One of the goals is to let them ``run the ship'' by the end of the course.}

\vfill


\noindent\textbf{Stagecraft} - \emph{Fall and Spring Semesters} \hfill Dane Peterson



{\small In this course, students learn the basics of set construction and scenery and lighting design. They build, paint, light a set and serve as stage crew for one major Indian Springs theater production during the semester. In addition, students set up and run lights, sound, and media for Town Meetings and other school functions on a rotating basis.}

\vfill

