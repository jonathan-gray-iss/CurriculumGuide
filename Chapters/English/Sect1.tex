\section{Portrait of a Graduate}

\textsc{An Indian Springs School graduate, having completed the course of study in English, will} $\ldots$ 
\begin{itemize}\item Understand and interpret visual texts (cartoons, sculpture, etc.).
\item Craft a visual response to texts that demonstrates understanding of the text’s rhetorical purpose.
\item Demonstrate guided mastery of database use.
\item Demonstrate mastery of close reading of texts.
\item Demonstrate mastery of parts of speech and how they are used.
\item Understand and adjust arguments for a variety of audiences.
\item Master reference text usage (Dictionaries, Encyclopedias, Journals, etc.).
\item Demonstrate mastery of various modes of writing.
\item Exhibit understanding of textual context – including, but not limited to culture, history of composition, history of the book/genre/form, biography of the author.
\item Master MLA citation form.
\item Be competent in assessing appropriateness of secondary sources.
\item Understand that reading, writing and thinking are concurrent processes.
\item Be able to ``read'' situations, as well as poems or short stories, and be able to ``confirm'' the ``text'' (discuss its purpose, tone, significance, argument, etc.) and ``complicate'' the ``text'' (question the author’s assumptions and linguistic/ rhetorical choices).
\item Communicate well to a variety of audiences.
\item Be independent, lifelong readers.
\item Be able to make a cogent, cohesive argument based on textual evidence.
\item Be able to research independently.
\item Be able to edit his or her own and others’ writing.
\item Recognize logical fallacies.
\item Have used personal essays as a vehicle for self-reflection.
\item Analyze and be able to criticize the author’s ``purpose'' with appropriate objectivity.
\item Write with subtlety and finesse.
\item Understand plagiarism and intellectual dishonesty.
\end{itemize}