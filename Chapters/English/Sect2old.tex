\section{Current Courses}

\noindent\textbf{English 8} \hfill Chow

\noindent Year - 1 Credit

\vspace{1mm}\emph{English 8 provides a strong foundation for further English courses at Indian Springs. Building on and reviewing basic grammar skills, students continue honing their sentence, paragraph and essay writing craft. Students read, analyze, speak and write about various types of literature, further developing their abilities to think and write effectively. Close reading, thorough examinations of diction, characterization, and tone plus spirited group discussions help students discover larger meanings, personal connections and an author's purpose in literary works from Ancient to Modern. Students also learn to identify major themes and recurring symbols important throughout literary /human history. Public Speaking/Oral Communication skills are also emphasized throughout the year through informal talks, formal speeches, and memorized Shakespearean monologues.}\\


\noindent\textbf{English 9} \hfill Barrett

\noindent Year - 1 Credit

\vspace{1mm}\emph{English 09 covers the fundamentals of writing and reading: writing claims, forming arguments, organizing paragraphs, syntax, and figurative language. Students will be expected to discuss and analyze major texts: The Teeth of the Comb by Osama Alomar, Daniel Quinn's Ishmael, Dante's Inferno, Edith Hamilton's Greek Mythology, Shakespeare's Romeo and Juliet, and Orson Scott Card's Ender's Game.  A central question that guides the curriculum is, ``How does mythology continue to shape our lives today?''}\\


\noindent\textbf{Critical Reading \& Analytical Writing} \hfill Griffin

\noindent Year - 1 Credit

\vspace{1mm}\emph{This yearlong course emphasizes critical reading and writing skills through the study of canonical and contemporary texts from around the globe. We will read novels, short stories, drama, creative nonfiction, and poetry written during the Renaissance to the present day, with particular emphasis on works produced during the twentieth and twenty-first centuries. We will study these texts as cultural records, which illuminate and offer commentary on the contexts from which they come. Additionally, this course emphasizes writing the literary analysis essay; making defensible, well-wrought arguments about a text in lucid, edited prose. Additionally, students will write and edit personal essays in order to search out their voices as writers and values as humans. Frequent informal writing will also take place in class. Finally, students will engage in a structured review of grammar, mechanics, and usage. }\\


\noindent\textbf{AP English Language and Composition} \hfill Woodruff

\noindent Year - 1 Credit

\vspace{1mm}\emph{AP English Language \& Composition covers the knowledge and skills of a college-level writing and rhetoric course (i.e. “freshman comp”). Students will enhance their critical thinking, reading, writing, listening, and speaking abilities. Writing will take center stage as students learn to identify, compare, critique, and produce arguments. Students will engage writing and research as process. The majority of readings will be non-fiction. Topics will often relate to ethics, politics, social issues, and/or language(s). The course likewise serves as an introduction to norms of communication and conduct in American post-secondary and professional settings. Students will thus learn and demonstrate dispositions conducive to success in those arenas.}\\


\noindent\textbf{Advanced Poetry II} \hfill Allen

\noindent Spring Semester - 0.5 Credits

\vspace{1mm}\emph{This is a writing and performance driven poetry class for grades 10-12 as an English elective (Fall 20223). Students will explore writing, performing, and teaching poetry as a powerful communicative outlet to unlock individual potential in establishing the writer’s voice. Performance master, John Paul Taylor is also featured in a few master classes. Lastly, there is an opportunity to correspond with other young students in order to teach poetry (pedagogy) and inspire others to write. Thus, potentially contributing to hosting several public exhibitions of youth poetry!}\\


\noindent\textbf{Major Authors: Salinger \& O'Connor} \hfill Woodruff

\noindent Spring Semester - 0.5 Credits

\vspace{1mm}\emph{Unlike a broad survey, this course will consider selected authors in depth. Specifically, we will study two of the most accomplished, complex, masterful, influential US fiction authors of the 20th century: J.D. Salinger and Flannery O’Connor. We will approach texts primarily from a formalist perspective (i.e. ``close reading'') informed by history, biography, and social concerns (e.g. race, gender, class, etc.).}\\


\noindent\textbf{Comparative Literature: Victorian Times} \hfill Woodruff

\noindent Spring Semester - 0.5 Credits

\vspace{1mm}\emph{A combination of survey and in-depth study of particular authors, this course will provide an introduction to 19th century British and French literature, art, and culture. We will consider the ways in which art forms in particular do and do not transcend formal, national, linguistic, spatial, and temporal boundaries. We and the Victorians will haunt ourselves and each other. All texts will be presented in English, though we will at times examine French ones in the original. French-speaking students will have opportunities to read, write, and speak French. We will approach texts primarily from formalist (i.e. “close reading”) and intertextual perspectives informed by history, biography, and social concerns (e.g. race, gender, class, etc.).}\\


\noindent\textbf{The Art of the Personal Narrative} \hfill Chow

\noindent Fall Semester - 0.5 Credits

\vspace{1mm}\emph{UNKNOWN}\\


\noindent\textbf{Literary Theory and Analysis} \hfill Chow

\noindent Spring Semester - 0.5 Credits

\vspace{1mm}\emph{UNKNOWN}\\


\noindent\textbf{The Graphic Novel} \hfill Allen

\noindent Spring Semester - 0.5 Credits

\vspace{1mm}\emph{By examining the ways in which each work assaults the status quo of an inhumane, often brutal society, we will develop the trajectory of the tradition of the Graphic Novel in literature and discover the means and methods of many writers from several different cultures and national literatures. We will connect these ideas to contemporary artistic expressions and developments within the media using film, graphic novels, music, poetry, and even viral campaigns.}\\


\noindent\textbf{Crime Fiction} \hfill Griffin

\noindent Fall Semester - 0.5 Credits

\vspace{1mm}\emph{Crime Fiction is a survey of the ever-popular genre of short stories and novels that cover everything from cozy whodunits to transgressive journeys into the psyches of the truly deranged. Along the way we will be reading from a number of essays, short stories and novels (and perhaps supplementing with an episode or film here and there) as we put on our deerstalker hats, pull out our meerschaum pipes and try to get to the bottom of the mystery of what it is about crime that continues to fascinate us.}\\


\noindent\textbf{Modern African-American Voices} \hfill Allen

\noindent Spring Semester - 0.5 Credits

\vspace{1mm}\emph{This is a writing/discussion intensive course for students 10-12 where we will engage upon an advanced study of critical theories of African-American literature through various contexts, including but not limited to the cultural criticism of 20th-century, ending at the conclusion of the American modern protest movement of the 1960s. Students will delve into a detailed study of African-American literature and its relationships to American culture and history, with an emphasis on fiction and poetry from 1780-1955. Such writers as Chesnutt, Dunbar, DuBois, Hughes, and Hurston will be explored at length.}\\


\noindent\textbf{Monsters, Devils, and Madmen} \hfill Barrett

\noindent Fall Semester - 0.5 Credits

\vspace{1mm}\emph{Monsters, Devils, and Madmen function as important literary tropes throughout literature.  They stand in for our own fears as well as our own hubris.  By examining how ``evil'' functions in works of classical literature throughout time, students will be asked to reevaluate how they imagine monsters in contemporary art.  Major texts include The Tragical History of Dr. Faustus by Christopher Marlow, Paradise Lost by John Milton, Dante's Inferno, Notes from the Underground by Fyodor Dostoevsky, and more.}\\


\noindent\textbf{Major Authors: Harlem Renaissance} \hfill Allen

\noindent Spring Semester - 0.5 Credits

\vspace{1mm}\emph{This is a writing/discussion intensive course for students where we will engage upon an advanced study of critical theories of African-American literature through various contexts, including but not limited to the cultural criticism of 20th-century. We will begin with Turn-of-the Century background readings, ending at the conclusion of the American modern protest movement of the 1930-40s. Students will delve into a detailed study of African-American literature and its relationships to American culture and history, with an emphasis on fiction and poetry since 1900. Such writers as Chesnutt, Dunbar, DuBois, Hughes, and Hurston will be explored at length. There may or may not be a surprise visit or two from Mr. Cal Woodruff.}\\


\noindent\textbf{Film Rhetoric} \hfill Griffin

\noindent Fall Semester - 0.5 Credits

\vspace{1mm}\emph{In this class we begin by examining the elements of film form: mise-en-scene, cinematography, editing, sound, narrative structure, and performance. Once introduced to these concepts, we examine together diverse films to hone your skills at analyzing how these elements build meanings. You will then practice applying these terms as you use them to describe and interpret films in both class discussion and written assignments.}\\


\noindent\textbf{Shakespearean Comedies \& Tragedies} \hfill Barrett

\noindent Fall Semester - 0.5 Credits

\vspace{1mm}\emph{Shakespeare is one of the world's most celebrated English writers.  This class is a survey of his tragedies/comedies.  Whether we read, listen to, watch, or discuss these plays, this class will expose students to major works within the Shakespearean canon.  Students will explore Shakespeare's biography, major themes, and relevant critical theory to write at least one major paper for the semester. }\\


\noindent\textbf{Romanticism} \hfill Barrett

\noindent Spring Semester - 0.5 Credits

\vspace{1mm}\emph{Romanticism is an ideological inheritance from the rise of industrialism.  The class will explore the philosophical principles of Romanticism while reading major Romantic texts: Hyperobjects by Timothy Morton, A Very Short Introduction to Romanticism by Michael Ferber, and the poetry of the British Romantics.  While appreciating the aesthetic contributions of Romantic art, students will also be asked to challenge dangerous Romantic presuppositions.  }\\


\noindent\textbf{Creative Writing Workshop} \hfill Barrett

\noindent Spring Semester - 0.5 Credits

\vspace{1mm}\emph{This is an opportunity for creative writers to hone their skills and develop a practice of writing, peer-review, editing and revising.  By examining successful short stories and poems, students will be expected to write throughout the semester and evaluate the work of their peers.  Tips on craft will be shared along the way and students will investigate literary journals for opportunities to publish their work.  Each student should end the semester with a writing portfolio.    }\\


\noindent\textbf{Contemporary African-American Voices=Renamed African-American Life and Lit Since 1968} \hfill Allen

\noindent Fall Semester - 0.5 Credits

\vspace{1mm}\emph{This is a writing/discussion intensive course where we will engage upon an advanced study of critical theories of African-American literature through various contexts, including but not limited to the cultural criticism of 20th-century Civil Rights movements through the post-modern protest movements of the 21st century. Students will delve into a detailed study of African-American literature and its relationships to American culture and history, with an emphasis on fiction and poetry since 1955. Much of this course features literature that focuses on major events, movements, and people in relation to the racial issues of the time period. African American authors focused on the “black experience,” and gave voice to protest against segregation and the reality of racial injustices. This literature centers on the conditional aspect of equality and the factors of race, gender, and ethnicity in eliminating racism and the accompanying social inequity. Many of these speakers and their writing do not fit the typical identity that students equate with the literature in the classroom. Some of the Black writers we will study are novelists, poets, and playwrights, but they are also journalists, lyricists/song writers, athletes and essayists. These writers and their work are prevalent and instrumental in modern American society}\\


\section{Past Courses}

\noindent\textbf{AP Literature and Composition}  - 1 Credit

\vspace{3mm}
\noindent\textbf{Outlaws, Outcasts, and Castaways}  - 0.5 Credits

\vspace{3mm}
\noindent\textbf{Epic Poetry-Spring}  - 0.5 Credits

\vspace{3mm}
\noindent\textbf{Epic Poetry--Fall}  - 0.5 Credits

\vspace{3mm}
\noindent\textbf{Strength, Struggle, and Staying the Course}  - 0.5 Credits

\vspace{3mm}
\noindent\textbf{Literary Genres:  The Comedy}  - 0.5 Credits

\vspace{3mm}
