\section{Current Courses}

\noindent\textbf{Science 8} \hfill Tetzlaff

\noindent Year - 1 Credit

\vspace{1mm}\emph{This course is a Physical Science course. In the first unit, students will apply scientific methodologies to solve problems and design experiments, learn metric measurement and dimensional analysis, collect and analyze data to reach conclusions, and explain their findings. The second unit is introductory chemistry. The students will explore properties of matter, atomic structure, the periodic table, chemical bonding, and chemical reactions. The third unit is introductory physics. The students will explore motion, forces, work, power, and machines, and energy. This is a laboratory-based science course.}\\


\noindent\textbf{Biology} \hfill Sides

\noindent Year - 1 Credit

\vspace{1mm}\emph{Cells are introduced as the structural and functional units of life, followed by an overview of the various kingdoms with emphasis on patterns in life histories, special adaptations, basic life processes, and sources of variation, with ecological and evolutionary implications.  In the spring, students study plant physiology, reproduction, and growth.  Additional topics include human anatomy, physiology, and modern genetics.}\\


\noindent\textbf{Chemistry} \hfill Hurt

\noindent Year - 1 Credit

\vspace{1mm}\emph{This year-long course introduces students to the basic concepts related to the study of matter.  Students conduct many laboratory experiments to enhance their knowledge of lab safety and techniques, and they write formal lab reports to further their ability to communicate scientific ideas and findings. Students learn the properties of matter, chemical formula and equation writing, stoichiometric calculations, gas laws, and bonding.}\\


\noindent\textbf{Conceptual Physics} \hfill Mohammed

\noindent Year - 1 Credit

\vspace{1mm}\emph{This course introduces students to topics of classical physics such as mechanics,
electricity and magnetism, optics, astronomy, and topics of modern physics such as atomic, nuclear, and
particle physics, and special relativity. Students develop their problem solving skills using algebra and
minimal trigonometry, but greater emphasis is placed on conceptual understanding. Students perform
laboratory experiments to enhance understanding of concepts, gain an appreciation for the process of
experimental science, and connect what they have learned to modern technology and careers in
science. Prerequisites: Biology, Chemistry, and Algebra II}\\


\noindent\textbf{AP Physics 1} \hfill Mohammed

\noindent Year - 1 Credit

\vspace{1mm}\emph{AP Physics 1 is an algebra-based, introductory college-level physics course.
Students cultivate their understanding of physics through classroom study, in-class activity, and hands-
on, inquiry-based laboratory work as they explore concepts like systems, fields, force interactions,
change, conservation, and waves. Prerequisites: Biology, Chemistry, and Algebra II. Pre-Calculus is strongly recommended but can be a corequisite.}\\


\noindent\textbf{AP Biology} \hfill Magnuson

\noindent Year - 1 Credit

\vspace{1mm}\emph{This course explores four major overarching themes in science: Evolution, Energetics, Information Storage and Transmission, and Systems Interactions. There are eight units that students will learn in order to discover the interrelatedness of the big ideas. Students will develop the following science skills through the course: Explaining Concepts, Analyzing Visual Representations, Determining Scientific Questions and Methods, Representing and Describing Data, Applying Statistical Tests and Data Analysis, and Developing and Justifying Scientific Arguments Using Evidence. This is a laboratory-based course that will allow students to strengthen current skills and explore new laboratory techniques. This course prepares students for the AP Exam and is equivalent to a year-long college Biology course. Prerequisites: Biology and Chemisty}\\


\noindent\textbf{AP Environmental Science} \hfill Magnuson

\noindent Year - 1 Credit

\vspace{1mm}\emph{This course is an interdisciplinary study of the interactions of organisms and the environment. The students will apply scientific principles, concepts, and methodologies to analyze and evaluate environmental problems and learn current practices and future areas of research proposed to remedy the problems. This is a laboratory and field study-based course that will allow students to strengthen current skills and explore new laboratory techniques.  This course prepares students for the AP Exam and is equivalent to a one-semester Environmental Science course. Prerequisites: Biology and Chemistry}\\


\noindent\textbf{Anatomy \& Physiology I} \hfill Hurt

\noindent Fall Semester - 0.5 Credits

\vspace{1mm}\emph{Anatomy and Physiology I is a lab and project based course that introduces students to the wonder of the human body. The course focuses on general anatomy including anatomical terminology, histology (looking at tissues under the microscope to understand the structure, and thus function of the tissue), integumentary system (skin) and skeletal system. Case studies are analyzed and diagnoses are justified with evidence.}\\


\noindent\textbf{Anatomy \& Physiology II} \hfill Hurt

\noindent Spring Semester - 0.5 Credits

\vspace{1mm}\emph{Anatomy and Physiology II focuses on the muscular, nervous, cardiovascular, and digestive systems. Dissections of sheep brains, eyes, and hearts, and a final rat dissection are completed in the course. Case studies are analyzed and diagnoses are justified with evidence.}\\


\noindent\textbf{AP Chemistry} \hfill Tetzlaff

\noindent Year - 1 Credit

\vspace{1mm}\emph{This course focuses on advanced studies in thermochemistry, oxidation-reduction reacations, equilibrium, reaction rates, electrochemistry, and kinetic molecular theory from the experimental and lab development perspectives. Students are prepared to take the AP Chemistry exam. Chemistry is a prerequisite for this course.}\\


\noindent\textbf{AP Physics C} \hfill Mohammed

\noindent Year - 1 Credit

\vspace{1mm}\emph{AP Physics C is equivalent to two one-semester, calculus-based, college-level
physics courses, especially appropriate for students planning to specialize or major in physical science or
engineering. The course explores topics such as kinematics; Newton’s laws of motion; work, energy and
power; systems of particles and linear momentum; circular motion and rotation; and oscillations and
gravitation as well as electric charge; electric fields; Gauss’ law; electric potential; capacitance; current
and resistance; circuits; magnetic fields; and induction and inductance. Introductory differential and
integral calculus is used throughout the course. Prerequisites: Biology, Chemistry, and Calculus (can be a corequisite)}\\


\noindent\textbf{Forensics} \hfill Magnuson

\noindent None - 0.5 Credits

\vspace{1mm}\emph{This course is a student-interest led class that covers (but is not limited to) crime scene investigation, evidence collection, fingerprint analysis, hair and fiber analysis, blood and blood spatter analysis, and urinalysis. Students will learn how to analyze and then perform the analyses, resulting in a lab-based course. This is a semester-long course.}\\


\noindent\textbf{Infectious Disease} \hfill Magnuson

\noindent None - 0.5 Credits

\vspace{1mm}\emph{This course explores immune system function, pathogens including bacteria and viruses, communicable diseases, and the emergence and current status of HIV. The students will learn sterile technique, culturing, staining, identifying bacteria with microscopy, and analyzing bacterial growth for antibiotic resistance. Students will evaluate case studies and propose solutions/course of treatment to solve. This is a lab-based course. This is a semester-long course.}\\


\noindent\textbf{Nutrition \& Metabolism} \hfill TBD

\noindent None - 0.5 Credits

\vspace{1mm}\emph{UNKNOWN}\\


\noindent\textbf{Forensics II} \hfill Magnuson

\noindent None - 0.5 Credits

\vspace{1mm}\emph{This course is a student-interest led class that covers (but is not limited to) toxicology, arson analysis, sutopsies, DNA profiling, and forgery.Students will learn how to analyze and then perform the analyses, resulting in a lab-based course. This is a semester-long course. Forensics is NOT a prerequisite for this course. }\\


\noindent\textbf{Molecular Genetics} \hfill Magnuson

\noindent None - 0.5 Credits

\vspace{1mm}\emph{This course will likely not be offered because the past content was unit 6 of AP Biology and had very little student interest.}\\


\noindent\textbf{Aquatic Ecology} \hfill Magnuson

\noindent None - 0.5 Credits

\vspace{1mm}\emph{This course will likely not be offered again because in it I covered marine and freshwater ecology and with the creation of Jeffrey's marine Biology there is too much redundancy.}\\


\noindent\textbf{Introduction to Marine Biology} \hfill Sides

\noindent None - 0.5 Credits

\vspace{1mm}\emph{UNKNOWN}\\


\noindent\textbf{Organic Chemistry} \hfill Tetzlaff

\noindent None - 0.5 Credits

\vspace{1mm}\emph{UNKNOWN}\\


\noindent\textbf{Psychology} \hfill Belser

\noindent Fall Semester - 0.5 Credits

\vspace{1mm}\emph{In this introductory course, we will be quickly covering many topics in psychology. It is a bold undertaking to try and cover 10 topics that range from Research Ethics and Learning Theory to Social Psychology in just 18 weeks, but we will do all we can to give each topic the attention that it deserves. We will have some guest speakers that are currently involved in studying and practicing psychology so students can get an idea of what a career in psychology looks like.}\\


\noindent\textbf{Exploration of the Microscopic World} \hfill Sides

\noindent None - 0.5 Credits

\vspace{1mm}\emph{UNKNOWN}\\


\noindent\textbf{Adv Topics in Chemistry} \hfill Tetzlaff

\noindent None - 0.5 Credits

\vspace{1mm}\emph{UNKNOWN}\\


\section{Past Courses}

\noindent\textbf{Sustainable Development - F}  - 0.5 Credits

\vspace{3mm}
\noindent\textbf{Geophysics}  - 0.5 Credits

\vspace{3mm}
\noindent\textbf{Astronomy of the Solar System}  - 0.5 Credits

\vspace{3mm}
\noindent\textbf{Experimental Procedures in Microbiology}  - 1 Credit

\vspace{3mm}
\noindent\textbf{Botany}  - 0.5 Credits

\vspace{3mm}
\noindent\textbf{Anthropology}  - 0.5 Credits

\vspace{3mm}
\noindent\textbf{Physics}  - 1 Credit

\vspace{3mm}
\noindent\textbf{Sustainable Development}  - 0.5 Credits

\vspace{3mm}