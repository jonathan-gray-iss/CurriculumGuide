\section{Current Courses}

\noindent\textbf{Intro to Computer Programming} \hfill Belser

\noindent Fall and Spring Semesters - 0.5 Credits

\vspace{1mm}\emph{This semester-long course will introduce students to the Java programming language, the NetBeans IDE, and the fundamental concepts in all computer programming languages. There are not any prerequisites for this class as it is an introduction from the very basics. We will make our way from a “Hello world.” program all the way to writing our own Tic Tac Toe game.}\\


\noindent\textbf{Intro to Engineering - Electronics} \hfill Belser

\noindent Spring Semester - 0.5 Credits

\vspace{1mm}\emph{A semester-long multidisciplinary class where students will learn the basics of electronics and circuitry. We will start with the basics concepts and simple designs and work our way to designing more complicated systems. We will create circuits that link sensors and other input devices to microcontrollers, that require some computer programming, to drive output devices such as LEDs and motors. By the end of the course, the goal is to be able to design and create a multi input and multi output system, usually an object avoidance vehicle}\\


\noindent\textbf{Linux} \hfill Belser

\noindent Fall Semester - 0.5 Credits

\vspace{1mm}\emph{Because the majority of server systems are Linux-based and universities have many of their math, statistical, deep learning, biological, genetic, and physical simulation systems housed on Linux/Unix servers, it is advantageous for students to have exposure and familiarity with these systems. This semester course will use the raspberry pi to introduce students to all aspects of a linux system so that they are comfortable using this server operating system.}\\


\noindent\textbf{AP Computer Science A} \hfill Belser

\noindent Year - 1 Credit

\vspace{1mm}\emph{After the Introduction to Computer Programming course, students can take this course to prepare to take the College Board Advanced Placement Computer Science A exam. We will go from creating simple games to understanding the intricacies of Object Oriented Programming and design strategies, problem solving methodologies, data structures, and algorithm design.}\\


\noindent\textbf{Big Data, Machine Learning, and AI} \hfill Belser

\noindent Spring Semester - 0.5 Credits

\vspace{1mm}\emph{This course starts with a basic introduction to the concepts and skills needed to manipulate and present data in an educational STEM environment. Using a programming environment known as ‘notebooks’, we have used the python programming language and data manipulation and presentation modules to build a quick report in Markdown and convert it to a PDF presentable report. It is OK for some of the students to have not python before or only have a cursory understanding of the language. These data skills are required before proceeding with tackling deep learning and artificial intelligence concepts and algorithms. Once mastered, the remainder of the course will be trying to see how far we can progress through the basic algorithms in deep learning and artificial intelligence.}\\


\section{Past Courses}

\noindent\textbf{Web Design}  - 0.5 Credits

\vspace{3mm}
\noindent\textbf{Python Programming}  - 0.5 Credits

\vspace{3mm}
\noindent\textbf{Intro to Engineering - 3D Design}  - 0.5 Credits

\vspace{3mm}
\noindent\textbf{Advanced Topics in Computer Science}  - 0.5 Credits

\vspace{3mm}