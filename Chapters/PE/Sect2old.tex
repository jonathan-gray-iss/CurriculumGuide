\section{Current Courses}

\noindent\textbf{8th Grade PE} \hfill Skiff/Van Horn

\noindent Year - 1 Credit

\vspace{1mm}\emph{Students in 8th grade PE will have the opportunity to experience physical activity in a fun and safe manner.  Putting the body “in motion” is important to good physical health and can be a great stress release when dealing with the demands of a school day.  The goals of the course are to have the students be physically active, to learn new activities and skills, and to build self confidence in the students.}\\


\noindent\textbf{9th Grade PE} \hfill Pino

\noindent Spring Semester - 0.5 Credits

\vspace{1mm}\emph{Students will take what they have learned in Wellness and Fitness and implement that knowledge into their daily activities.  The focus will be working towards personal fitness goals through group and individual activities.}\\


\noindent\textbf{Well/Fit} \hfill Pino

\noindent Fall Semester - 0.5 Credits

\vspace{1mm}\emph{Wellness and Fitness is an introductory course dedicated to promoting a lifestyle which results in total health and wellness. The course is composed of both classroom and gym days consisting of personal assessment, taking notes, and physical activity.  Popular topics discussed include cardiovascular and muscular strength exercise, nutrition, and stress management.}\\


\noindent\textbf{Foundation of Sports Medicine and Safety} \hfill Skiff

\noindent Fall Semester - 0.5 Credits

\vspace{1mm}\emph{This course is designed to introduce the ideas and concepts that surround the growing field of sports medicine.  Students will explore the relationship of risk management and injury prevention through those fields that are defined as sports medicine. Students will examine the sports medicine team, sports medicine facilities, policies, procedures, and protocols utilized in patient care. Emphasis will be placed on health promotion, athlete wellness, and injury and disease prevention within athletic groups. Weekly discussions on current injured athletes will be highlighted.  This is a prerequisite to Sports Medicine I.}\\


\noindent\textbf{Injury Prevention \& Weight Training} \hfill Skiff

\noindent Fall and Spring Semesters - 0.5 Credits

\vspace{1mm}\emph{This course is designed to introduce the ideas and concepts that surround the prevention of injuries utilizing sound principles of weight training.  Students will learn the fundamentals of body assessment, establishing a physical foundation, and how to design a basic weight training/conditioning program specific to their needs.  Students will be able to identify general anatomy, weight room safety, correct weight lifting techniques, setting appropriate goals, yoga/stretching, and core work.}\\


\noindent\textbf{Sports Medicine} \hfill Skiff

\noindent Spring Semester - 0.5 Credits

\vspace{1mm}\emph{After establishing the basic understanding of the sports medicine team in Foundations of Sports Medicine and Safety, students will take a comprehensive look at the upper and lower extremities of the body.  Starting with the head and working their way down to the feet, the students will learn the anatomy, evaluation of the most common injuries, and basic rehabilitation of each body part.  Interactive labs will be introduced to include taking vitals, preventative taping, rehabilitation practices and the use of therapeutic modalities for the most common sports medicine injuries.}\\


\noindent\textbf{10th Grade PE} \hfill Van Horn

\noindent Year - 1 Credit

\vspace{1mm}\emph{Students in the 10th and 11th grade are required to find/develop a physical activity to meet their PE requirement of active exercise resulting in a minimum of 2-3 hours per week.  Physical activities may include any sport offered at Indian Springs School, other activities offered at Indian Springs School such as intramurals, or an outside-of-school program that involves physical exercise.  Any student-designed program must have a supervising instructor, meet two of the three components of fitness (cardiovascular endurance, muscular endurance and flexibility), and meet the approval of the PE department prior to the start of the program.}\\


\noindent\textbf{11th Grade PE} \hfill Van Horn

\noindent Year - 1 Credit

\vspace{1mm}\emph{Students in the 10th and 11th grade are required to find/develop a physical activity to meet their PE requirement of active exercise resulting in a minimum of 2-3 hours per week.  Physical activities may include any sport offered at Indian Springs School, other activities offered at Indian Springs School such as intramurals, or an outside-of-school program that involves physical exercise.  Any student-designed program must have a supervising instructor, meet two of the three components of fitness (cardiovascular endurance, muscular endurance and flexibility), and meet the approval of the PE department prior to the start of the program.}\\


