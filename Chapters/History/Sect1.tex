\section{Portrait of a Graduate}

\textsc{An Indian Springs School graduate, having completed the course of study in History, will} $\ldots$ 

\begin{itemize}


  \item Be prepared for a future of lifelong learning and active, responsible global citizenship.
\begin{itemize}
  \item Recognize that individuals are agents of historical change and that an individual today can be an engaged and informed citizen who affects change in the world.

  \item Understand the differences between major forms of political, economic, and social organization across times and places
  \item Theorize and practice the fundamentals of individual and group self-governance
  \item Have developed strategies to maximize their own learning strengths, including digital and civic literacy necessary to navigate the modern world of information and mis/disinformation  

\end{itemize}
  \item Possess an historical perspective of who they are, why the world is the way it is, and how the past systemically influences the present.

  \item Grasp the interconnectedness of geography, politics, economics, social conditions, and ideas; the role of power in each; and the impact of each on the human experience.  
\begin{itemize}

      \item Think critically, including:

      \item Assimilate and synthesize large amounts of information,
      
      \item Evaluate the credibility and limitations of evidence and arguments
      
      \item Construct and defend theories of the human condition, such as political theory, social theory, and theory of mind
      
      \item Analyze and interpret historical documents
      \item Problem solving.
\end{itemize}
  \item Think historically about relationships/connections (comparison, causation, contextual) in the human experience.  
  \begin{itemize}
    \item Think chronologically and explain continuity and change. 
    \item Be able to draw comparisons between time periods and regions in order to identify transcending themes. 
    \item Be able to analyze cause and effect, including multiple causation, and to challenge arguments of inevitability.
    \item Be able to compare and contrast competing historical narratives and evaluate major debates among historians. 
  \end{itemize}
  \item Communicate effectively, including reading comprehension, writing, speaking, and listening.
  \begin{itemize}


    \item Be able to create and support contestable thesis statements
    \item Be able to structure and support logical argument.
    \item Be able to interpret and deconstruct the arguments of others
  \end{itemize}
  
  \item Be able to conduct effective research.
  \begin{itemize}


    \item Be comfortable with independent learning.
    \item Be able to formulate historical questions. 
    \item Be able to obtain and evaluate data (both primary and secondary sources), consider gaps in what we know, and use data to support an argument.
    \item Understand the difference between one’s own original thought and someone else's.
    \item Be competent with MLA and Chicago styles. 
  \end{itemize}
  \item Be sensitive to bias and understand that the present influences our understanding and interpretation of the past.

\begin{itemize}
  \item Be able to distinguish between different forms of bias and understand how points of view shape people’s interpretations of events and ideas
\end{itemize}
\end{itemize}