\section{Current Courses}

\noindent\textbf{Eighth-Grade Social Studies} \hfill Cooper

\noindent Year - 1 Credit

\vspace{1mm}\emph{The major goal of the course is to produce lifelong learners who are “yearning for learning.”  The instructor will attempt to instill enthusiasm for learning through class discussion, where the preconceptions of the students are regularly challenged, through interesting and challenging reading assignments, and through writing assignments that demand thoughtful analysis, logical organization, and competent writing skills.  Improved communications skills are a key course goal, along with the development of critical thought processes, a knowledge of geography, and an awareness of current global issues.}\\


\noindent\textbf{World History to 1200} \hfill Jacobs

\noindent Year - 1 Credit

\vspace{1mm}\emph{This course explores major events in the development of world history from the paleolithic world through the start of High Period. The approach is interdisciplinary and thematic, emphasizing political, economic, social, philosophical, scientific, literary, and artistic interrelationships across time and place. Twenty-first century skills such as problem solving, information literacy, and critical thinking are stressed.}\\


\noindent\textbf{AP World History} \hfill Clinkman

\noindent Year - 1 Credit

\vspace{1mm}\emph{A study of the major political, economic, and social events and fundamental themes of world history over the last five centuries, as well as the social, cultural, and intellectual movements that precipitated or were inspired by those events. Course discussions will center on the narrative of world history as well as major themes that have arisen over time, constantly reiterating the interconnectedness of different time periods.}\\


\noindent\textbf{AP United States History} \hfill TBD

\noindent Year - 1 Credit

\vspace{1mm}\emph{This course traces the history of the United States from its colonial origins in the late sixteenth century to the 1980s. Through common readings, discussions, and lectures, students explore the distinctive rhythms (political, economic, and social) of the American historical experience. }\\


\noindent\textbf{Early Modern European History} \hfill Clinkman

\noindent Fall Semester - 0.5 Credits

\vspace{1mm}\emph{This course builds upon prior coursework in AP World History to go into greater depth on the history of Europe during the early modern period (c.1400-1800). Primary emphasis will be upon the major intellectual movements of this period - the Renaissance, Reformation, Scientific Revolution, and Enlightenment - but students will also engage with the political and social history that contextualized those movements, such as the Wars of Religion and the French Revolution. Students will be assessed via a mix of written work and participation.}\\


\noindent\textbf{Modern European History} \hfill Clinkman

\noindent Spring Semester - 0.5 Credits

\vspace{1mm}\emph{This course builds upon Early Modern European History and AP World History to go into greater depth on the history of Europe during the modern period (c.1800-present). Primary emphasis will be upon the major political movements of this time period, with special focus on the conflict between liberalism, communism, and fascism that dominated the twentieth century. Students will be assessed via a mix of written work and participation.}\\


\noindent\textbf{Cooper Seminar} \hfill Cooper

\noindent Fall and Spring Semesters - 0.5 Credits

\vspace{1mm}\emph{Each semester Dr. Cooper leads a seminar to investigate some area of the social sciences. Past seminars have included such subjects as Global Issues, U.S. Issues, the French Revolution, the Warrior in History, Reconstruction after the Civil War, the Civil War, Perspectives on Death, Russian history, Tolstoy’s Philosophy of History, Intellectual History, Philosophies of Education, Diplomatic History, etc. Classes are structured around class discussion and assessments consist of four major papers that can be rewritten as often as the student deems necessary. The goals of the seminar, apart from encouraging the mastery of the material, are the development of critical thinking and the improvement of communication skills.}\\


\noindent\textbf{American Government} \hfill Wainwright

\noindent Fall Semester - 0.5 Credits

\vspace{1mm}\emph{This course provides an introduction to the philosophical foundations of the Constitution and its main themes—popular sovereignty, separation of powers, federalism, civil rights/liberties; an overview of the nature, structure, and functions of government institutions and how they operate in a system of limited powers; and the basics of American politics, including discussion of political parties, voting rights, the electoral process, interest groups, public opinion, social movements, and the media.}\\


\noindent\textbf{Constitutional Law \& Civil Rights} \hfill Wainwright

\noindent Fall Semester - 0.5 Credits

\vspace{1mm}\emph{This course provides an introduction to Constitutional Law and the legal doctrines that lawyers and judges use to analyze the Constitution’s protection of civil rights, with a particular focus on equality, as determined by the Equal Protection Clause (focusing particularly on race and gender discrimination, as well as “affirmative action” in employment and college admissions); privacy rights protected under the Due Process Clause (including reproductive autonomy, marriage, and sexual orientation/sexual activity); and voting rights, from the Voting Rights Act of 1965 to modern voter suppression tactics.}\\


\noindent\textbf{History of American Democracy} \hfill Wainwright

\noindent Spring Semester - 0.5 Credits

\vspace{1mm}\emph{Modeled after the Harvard class of the same name, this course uses the “case method,” a teaching method developed at Harvard Business School. The syllabus is built around twenty case studies on key episodes in American history, each ending at a pivotal moment in U.S. history and raising questions that faced key decision makers at the time. Students are put in the role of those decision makers and are left to wrestle with and resolve those questions, both on their own and in the classroom. One of the virtues of the case method is its ability to encourage the spirit of deliberative problem solving—which is, after all, at the heart of democracy. The use of a protagonist in each case study—that is, the individual or group whose role the students are asked to step into—demands that students ask themselves two related questions: “What would this protagonist do?” and “What would I do?” Active participation in class is essential to the method, and the grading reflects its importance.}\\


\noindent\textbf{International Relations} \hfill Clinkman

\noindent Fall Semester - 0.5 Credits

\vspace{1mm}\emph{International relations (IR) is an interdisciplinary and exciting field that brings together political science, economics, and cultural studies. In this class, we will engage with formal IR theory before embarking on specific studies in the areas of national security, political economy, and international integration. Students will be provided the opportunity to engage with IR both through written work and enacting simulations of IR scenarios.}\\


\noindent\textbf{Moral Philosophy} \hfill Clinkman

\noindent Spring Semester - 0.5 Credits

\vspace{1mm}\emph{Is morality absolute or relative? Is there a greater expectation of perfection or progress? What do we owe to each other? This course will explore frameworks for ethical thinking, using NBC’s The Good Place as a study paradigm. Students will work in small groups as well as participate in class discussions responding to the major ideas introduced in The Good Place, with further exploration of key concepts through primary and secondary sources. The course will be assessed through a combination of written work and participation.}\\


\noindent\textbf{Introduction to Economics} \hfill Wolfe

\noindent Fall and Spring Semesters - 0.5 Credits

\vspace{1mm}\emph{The focus of this course is on the basic principles concerning production, consumption, and distribution of goods and services in the United States and a comparison to other countries around the world. Students will examine the rights and responsibilities of consumers and businesses. Students will analyze the interaction of supply, demand, and price and study the role of financial institutions in a free enterprise system. Types of business ownership and market structures are discussed, as are basic concepts of consumer economics. The impact of a variety of factors including geography, the federal government, economic ideas from important philosophers and historic documents, societal values, and scientific discoveries and technological innovations on the national economy and economic policy is an integral part of the course. Students will apply critical-thinking skills to create economic models and to evaluate economic-activity patterns. We will also delve into personal finances as it relates to basic economic principles to develop a better understanding of money and how it functions in society.}\\


\noindent\textbf{Business Entrepreneurship} \hfill Wolfe

\noindent Spring Semester - 0.5 Credits

\vspace{1mm}\emph{This is a follow-up course to Introduction to Economics where students will learn about business entrepreneurship. We will discuss the underlying principles of starting a business, how to avoid common pitfalls, how to pitch ideas more effectively, how to validate your product to the market, how to develop a solid business model, and how to set up for success in a field where failure is common. Students will apply critical-thinking skills to create a business plan to pitch to investors at the end of the semester. The final project will start from a business idea to form a business plan to take that plan and then form a pitch deck to share with "investors" at the end of the semester.}\\


\noindent\textbf{Feminist Theory} \hfill Jacobs

\noindent Fall Semester - 0.5 Credits

\vspace{1mm}\emph{Feminist theory is a major branch of sociology that focuses on sex and gender, structural and economic inequalities, and power and oppression.  Feminist thought has a history often imagined as waves and a set of labels including liberal, radical, Marxist, intersectional, psychoanalytic, postmodern, and more.  These labels serve as the thematic units for this course.  Required textbook readings are supplemented with articles, lectures, music, art, case law, and other sources.  For each unit, students will complete minor assignments, and for four units, students will be required to complete a major assessment from a menu of options including a traditional essay on a major author, traditional essay topical analysis, spotify playlist with liner notes, analysis of a film/series, or analysis of a major work/project from another class.  Other assessments will be considered with prior instructor approval.}\\


\noindent\textbf{Religious Literacy} \hfill Jacobs

\noindent Spring Semester - 0.5 Credits

\vspace{1mm}\emph{In 1966, Time Magazine (for the first time ever) published a cover with no picture – only text. It was a solid black background with bright red letters that asked, “Is God Dead?” In the mid-1960s, the country was reeling with dramatic social change, and part of that shift was a new secularization that got many people wondering if the US was on the path to becoming post-religious. The past few years have made it clear that we are not post-religious, that religion remains a big piece of our national political discussion and a big part of our national identity. Yet Americans on the whole are woefully ignorant about religion. In an effort to separate church and state and in attempts to communicate with those different from us, we somehow got it in our minds that religion (like politics and football) are things we politely don’t discuss.}\\