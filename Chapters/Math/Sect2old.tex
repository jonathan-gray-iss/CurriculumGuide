\section{Current Courses}

\noindent\textbf{Foundations in Algebra and Geometry} \hfill TBD

\noindent Year - 1 Credit

\vspace{1mm}\emph{This first-year course prepares students for the challenges ahead in the math curriculum. Students examine and represent numbers in various forms; demonstrate fluency in mathematical language and understanding of concepts, processes, and reasoning; develop independence in learning mathematics; investigate math's scope and nature; and acquire a broad yet solid foundation for both algebra and geometry. They apply their learning to an array of problems.}\\


\noindent\textbf{Algebra I} \hfill TBD

\noindent Year - 1 Credit

\vspace{1mm}\emph{Algebra 1 is a course that introduces basic algebraic skills and focuses on problem solving techniques. It is designed as an introductory high school math course to provide a foundation for all subsequent math courses. Students must have a solid understanding pre-algebra topics, learn to think abstractly and become proficient problem solvers. Topics include, but are not limited to, properties of real numbers, relations and functions, linear equations and functions, Absolute value equations and functions, linear inequalities, systems of equations, properties of exponents, quadratic expressions and equations, radical expressions and equations, and rational expressions and equations. All units will contain graphical analysis, relating graphs to their corresponding expression or equation. The pace and depth of this course distinguishes it from Advance Algebra 1.}\\


\noindent\textbf{Advanced Algebra I} \hfill TBD

\noindent Year - 1 Credit

\vspace{1mm}\emph{Advanced Algebra 1 is a rigorous course that introduces basic algebraic skills and provides the foundation for all subsequent math courses. It is designed for students who have demonstrated exceptional ability and motivation in mathematics. Students must be highly motivated with a solid understanding pre-algebra topics, be able to think abstractly and be proficient problem solvers. Topics include, but are not limited to, properties of real numbers, relations and functions, linear equations and functions, Absolute value equations and functions, linear inequalities, systems of equations, properties of exponents, quadratic expressions and equations, radical expressions and equations, and rational expressions and equations. All units will contain graphical analysis, relating graphs to their corresponding expression or equation.}\\


\noindent\textbf{Geometry} \hfill TBD

\noindent Year - 1 Credit

\vspace{1mm}\emph{Geometry is a foundational course focused on the geometry of shapes, planes and space. Emphasis is placed on understanding, applying, justifying, and developing geometric properties in two and three dimensions. Students will engage in an in depth study of geometric reasoning, coordinate geometry, parallel and perpendicular lines, triangles, quadrilaterals, properties of polygons and circles, congruence and similarity, constructions, right triangle trigonometry, area, and volume. Students will apply this learning to solve real-world mathematical problems. The pace and depth of this course distinguishes it from Advance Geometry.}\\


\noindent\textbf{Advanced Geometry} \hfill TBD

\noindent Year - 1 Credit

\vspace{1mm}\emph{Advanced Geometry is a foundational course focused on the geometry of shapes, planes and space. Emphasis is placed on understanding, applying, justifying, and developing geometric properties in two and three dimensions. Students will engage in an in depth study of geometric reasoning, coordinate geometry, parallel and perpendicular lines, triangles, quadrilaterals, properties of polygons and circles, congruence and similarity, constructions, right triangle trigonometry, area, and volume. Students will apply this learning to solve real-world mathematical problems.}\\


\noindent\textbf{Algebra II w/ Trigonometry} \hfill TBD

\noindent Year - 1 Credit

\vspace{1mm}\emph{Algebra II with Trigonometry is designed for students who have progressed the typical sequence of mathematics. It introduces students to advanced functions, with a focus on developing a strong conceptual grasp of the expressions that define them. Students learn through discovery and application, developing the skills they need to break down complex challenges and demonstrate their knowledge in new situations. To be successful students must complete daily work and be disciplined to read, listen, and think independently. Course topics include, but are not limited to, a complete study of function including quadratic functions, polynomial functions, rational functions, radical functions, exponential and logarithmic functions, and trigonometric functions. The pace and depth of this course distinguishes it from Advance Algebra 2.}\\


\noindent\textbf{Advanced Algebra II w/ Trigonometry} \hfill TBD

\noindent Year - 1 Credit

\vspace{1mm}\emph{Advanced Algebra II with Trigonometry is designed for students who have demonstrated exceptional ability and motivation in mathematics. It introduces students to advanced functions, with a focus on developing a strong conceptual grasp of the expressions that define them. Students learn through discovery and application, developing the skills they need to break down complex challenges and demonstrate their knowledge in new situations. Students must be highly motivated with a solid understanding of previous math courses, be able to think abstractly and be proficient problem solvers. There is a rapid progression of topics and students must be able to perform within time limits. To be successful students must complete daily work and be disciplined to read, listen, and think independently. Course topics include, but are not limited to, a complete study of function including quadratic functions, polynomial functions, rational functions, radical functions, exponential and logarithmic functions, and trigonometric functions.}\\


\noindent\textbf{College Algebra} \hfill TBD

\noindent Year - 1 Credit

\vspace{1mm}\emph{College Algebra and Statistics is a year-long course and is equivalent to an introductory college math course. This course is a functional approach to algebra that incorporates the use of appropriate technology. Emphasis will be placed on the study of functions and inequalities, including their graphs. Functions to be studied are linear, quadratic, piece-wise defined, absolute value, rational, polynomial, radical, exponential, logarithmic, and trigonometric functions. Appropriate applications will be included. In addition, this course will introduce students to the study of Statistics. Topics from Statistics to be covered are, but not limited to, numerical and graphical data analysis, probability, the Normal, Binomial, and Geometric Distributions, and simple linear regression.}\\


\noindent\textbf{Advanced Precalculus} \hfill TBD

\noindent Year - 1 Credit

\vspace{1mm}\emph{Advanced Precalculus is designed for the student who has a high interest in math or areas related to math. It builds on and completes the advanced concepts began in Advanced Algebra 1, Advanced Geometry and Advanced Algebra 2. Topics include, but are not limited to, polynomial and rational functions, complex numbers, determinants, inverse functions, trigonometry, logarithms, and exponentials. It introduces vectors, polar coordinates, parametric equations, matrix theory, partial fractions, limits, and some basic operations of calculus. }\\


\noindent\textbf{AP Calculus AB} \hfill TBD

\noindent Year - 1 Credit

\vspace{1mm}\emph{AP Calculus AB focuses on students’ understanding of calculus concepts and provides experience with methods and applications. Through the use of big ideas of calculus (e.g., modeling change, approximation and limits, and analysis of functions), the course becomes a cohesive whole, rather than a collection of unrelated topics. The course requires students to use definitions and theorems to build arguments and justify conclusions. The course features a multirepresentational approach to calculus, with concepts, results, and problems expressed graphically, numerically, analytically, and verbally. Exploring connections among these representations builds understanding of how calculus applies limits to develop important ideas, definitions, formulas, and theorems. A sustained emphasis on clear communication of methods, reasoning, justifications, and conclusions is essential. Teachers and students should regularly use technology to reinforce relationships among functions, to confirm written work, to implement experimentation, and to assist in interpreting results. This course follows the syllabus of the Advanced Placement Calculus AB exam. It is equivalent to one semester of college Calculus.}\\


\noindent\textbf{AP Statistics} \hfill TBD

\noindent Year - 1 Credit

\vspace{1mm}\emph{The AP Statistics course introduces students to the major concepts and tools for collecting, analyzing, and drawing conclusions from data. There are four themes evident in the content, skills, and assessment in the AP Statistics course: exploring data, sampling and experimentation, probability and simulation, and statistical inference. Students use technology, investigations, problem solving, and writing as they build conceptual understanding. This course follows the syllabus of the Advanced Placement Statistics Exam. This course is equivalent to an introductory Statistics in college.}\\


\noindent\textbf{AP Calculus BC} \hfill TBD

\noindent Year - 1 Credit

\vspace{1mm}\emph{AP Calculus BC focuses on students’ understanding of calculus concepts and provides experience with methods and applications. Through the use of big ideas of calculus (e.g., modeling change, approximation and limits, and analysis of functions), the course becomes a cohesive whole, rather than a collection of unrelated topics. The course requires students to use definitions and theorems to build arguments and justify conclusions. The course features a multirepresentational approach to calculus, with concepts, results, and problems expressed graphically, numerically, analytically, and verbally. Exploring connections among these representations builds understanding of how calculus applies limits to develop important ideas, definitions, formulas, and theorems. A sustained emphasis on clear communication of methods, reasoning, justifications, and conclusions is essential. Teachers and students should regularly use technology to reinforce relationships among functions, to confirm written work, to implement experimentation, and to assist in interpreting results. This course follows the syllabus of the Advanced Placement Calculus BC exam. It is equivalent to two semesters of college Calculus.}\\


\noindent\textbf{Differential Equations} \hfill TBD

\noindent Fall Semester - 0.5 Credits

\vspace{1mm}\emph{Introduces ordinary differential equations by means of algebraic, numerical, and graphical analysis (including phase-plane analysis). Examines first order differential equations, second and higher order linear equations, methods for nonhomogeneous second order equations, series solutions, Laplace transforms, linear systems, and linearization of nonlinear systems. Covers various applications throughout the course. Requires a graphing calculator with the TI-84 Plus series recommended. Students in this course will use the skills learned in calculus extensively. AP Calculus BC is a prerequisite.}\\


\noindent\textbf{Discrete \& Combinatorial Math} \hfill Gray

\noindent Spring Semester - 0.5 Credits

\vspace{1mm}\emph{One could say discrete mathematics is the study of the properties of the integers. In recent times, the importance of the field has been proven because computers work in a discrete manner (bits) and various mathematical structures can be used to represent theoretical models in computer science. In this course we will meander through various topics in discrete mathematics and step outside the standard curriculum to study knot theory (classification, invariants, knot polynomials). The standard curriculum will include set theory (naive, functions, injectivity, surjectivity, enumerating functions), combinatorics (permutations, combinations, complementary counting, symmetry, combinatorial proofs, 12-fold way), sequences (arithmetic/geometric sequences/sums, polynomial fitting, recurrence relations, characteristic root technique, induction), calculus of finite differences (factorial polynomials, fundamental theorem, antidifferences), number theory and group theory (divisibility, modular arithmetic, equations in Z/nZ, structure of Z/nZ for n prime, Chinese remainder theo- rem), graph theory (planarity, coloring, paths, circuits, bipartite graphs, incidence matrices), and apportionment.}\\


\noindent\textbf{Linear Algebra} \hfill Gray

\noindent Fall Semester - 0.5 Credits

\vspace{1mm}\emph{A standard treatment of linear algebra as presented to university-level mathematics majors. Course topics will include row-reduction, matrix equations, linear transformations, matrix opera- tions, invertibility, LU-factorization, subspaces of Euclidean space, dimension, rank, determinants (elementary product definition, expansion by minors, and row-reduction), vector spaces, null and column spaces, linear independence, bases, change of basis, eigen-theory, algebraic and geometric multiplicity, diagonalization, inner product, length, orthogonality, orthogonal sets, projections, the Gram-Schmidt process, QR-factorization, and the method least-squares.}\\


\noindent\textbf{Multivariable Calculus} \hfill TBD

\noindent Spring Semester - 0.5 Credits

\vspace{1mm}\emph{Multivariable Calculus is a college-level course that follows Advanced Placement Calculus BC. The course emphasizes a thorough study of vectors, surfaces in space, vector-valued functions, functions of several variables, multiple integrations, and vector analysis. Students will become proficient at vector operations including the dot product and cross product and their applications, rectangular coordinates, cylindrical coordinates, and spherical coordinates. Students will learn operations and applications of vector-valued functions including differentiation, integration, velocity, acceleration, tangent vectors, and normal vectors. Realizing that many real-life quantities are functions of two more variables, students will understand the following implementations of functions of several variables: limits, continuity, derivatives, and integration. The goal is to learn, understand, and be able to work with the main ideas of multivariable calculus.}\\


