\section{Course Descriptions}

\subsection{Current Courses}
\noindent\textbf{Chinese I} \hfill Chang

\noindent Year - 1 Credit

\vspace{1mm}\emph{The course begins with learning the Basic Four as the foundation of learning Chinese: Pinyin, Tones, strokes, and Radicals. The lessons include Numbers, Greetings, Dates, Times, Phones, Family, Profession, Colors, etc. Learners can understand basic language materials related to common daily settings. Learners can repeat, recite, and reproduce words or sentences with fair accuracy. When students have studied Chinese for one semester or an academic year, with 4-5 class hours each week, these students have mastered 150 commonly used words and basic grammar patterns. (Novice - Novice/Mid).}\\

\noindent\textbf{Chinese II} \hfill Chang

\noindent Year - 1 Credit

\vspace{1mm}\emph{The course begins with learning the Countries and Languages, School Subjects, Weather, Seasons, Sickness, Meals, House, Community, etc. Learners start to have preliminary knowledge of learning, communication, resource, and interdisciplinary study under guidance.  Learners will gain introductory Chinese cultural understanding and acquire primary cross-cultural awareness and an international perspective. When students have studied Chinese for one to two academic years or more, with 4-5 class hours each week, these students have mastered 300 commonly used words and basic grammar patterns. (Novice Mid - Novice High)}\\

\noindent\textbf{Chinese III} \hfill Chang

\noindent Year - 1 Credit

\vspace{1mm}\emph{The course begins with Appearance, Seeing a Doctor, Occupation, Sports, Examinations, School Facilities, asking for Directions, Neighbors, etc. Learners will gain introductory Chinese cultural understanding and acquire primary cross-cultural awareness and an international perspective. When students have studied Chinese for two to three academic years or more, with 4-5 class hours each week, these students have mastered 600 commonly used words and basic grammar patterns. (Novice High - Intermediate Low)}\\

\noindent\textbf{Chinese IV} \hfill Chang

\noindent Year - 1 Credit

\vspace{1mm}\emph{The course begins with learning the Personalities, Daily Routine, Household Chores, School & Class Schedules, Career, Gourmet, Tourism, etc. Learners can understand the related language materials on social life and produce more correct sentences on familiar topics in description, explanation, or comparison.  Learners can compose a simple paragraph or essay by demonstrating confidence and interest in learning the Chinese language.  Learners can master specific knowledge of strategies for learning, communication, resource, and interdisciplinary study.  Learners can gain introductory Chinese cultural understanding and acquire preliminary cross-cultural awareness and an international perspective. When students have studied Chinese for four academic years or more, with 4-5 class hours each week, these students have mastered 1200 commonly used words and basic grammar patterns.     (Intermediate Low - Intermediate High)}\\



\noindent\textbf{Chinese V} \hfill Chang

\noindent Year - 1 Credit

\vspace{1mm}\emph{	The course begins with Making Friends, Buying Plane Tickets, Shopping, Learning Chinese, Summer Jobs, Food, Accidents, Volunteering, Chinese New Year, etc. Learners can understand a wide range of topics, produce correct sentences, write in paragraphs, create cohesive discourse, and express themselves fluently and spontaneously without much obvious searching for expressions. Learners have mastered the knowledge of learning strategies, resource strategies, and interdisciplinary strategies. Learners can gain extensive Chinese cultural knowledge and acquire preliminary cross-cultural awareness and international perspectives. When students have studied Chinese for five academic years, with 4-5 class hours each week, these students have mastered 2500 commonly used words and basic grammar patterns. (Intermediate High - Advanced)}\\



\noindent\textbf{French I} \hfill Bassene

\noindent Year - 1 Credit

\vspace{1mm}\emph{During the first two years, students acquire a basic proficiency in speaking, listening, reading, and writing.  An interactive video program rich in cultural content serves as a basis for class discussions which are conducted mostly in French.}\\

\noindent\textbf{French II} \hfill Bassene

\noindent Year - 1 Credit

\vspace{1mm}\emph{During the first two years, students acquire a basic proficiency in speaking, listening, reading, and writing.  An interactive video program rich in cultural content serves as a basis for class discussions which are conducted mostly in French.}\\

\noindent\textbf{French III} \hfill Bassene

\noindent Year - 1 Credit

\vspace{1mm}\emph{Students complete the acquisition of basic communication skills and begin to study literature, which provides them with opportunities for the analysis of content and style through a variety of written and spoken activities.  A grammar text supplements the course material as the students refine their control of the linguistic structures of French.}\\

\noindent\textbf{French IV} \hfill Bassene

\noindent Year - 1 Credit

\vspace{1mm}\emph{In the advanced levels of French, students continue their literary studies and are expected to strengthen their language skills through more in-depth class discussions, oral presentations, compositions, and the regular engagement of the French-speaking world through the use of authentic materials.  In these classes, students and teachers engage in informative conversations that range from micro-cultural studies of various Francophone localities to the often problematical and nuanced geo-political realities—past, present, and future.}\\

\noindent\textbf{AP French Language \& Culture} \hfill Bassene

\noindent Year - 1 Credit

\vspace{1mm}\emph{In the advanced levels of French, students continue their literary studies and are expected to strengthen their language skills through more in-depth class discussions, oral presentations, compositions, and the regular engagement of the French-speaking world through the use of authentic materials. In these classes, students and teachers engage in informative conversations that range from micro-cultural studies of various Francophone localities to the often problematical and nuanced geo-political realities\&mdash;past, present, and future. Students in Level V may elect to take the AP French Language exam. Students who have completed Level V prior to their senior year can receive instruction at Level VI.}\\

\noindent\textbf{Latin I} \hfill Crowe

\noindent Year - 1 Credit

\vspace{1mm}\emph{In this introductory course, students study basic vocabulary and grammar, Greek and Roman history, and the influence of the Latin language upon the English language.}\\

\noindent\textbf{Latin II} \hfill Crowe

\noindent Year - 1 Credit

\vspace{1mm}\emph{This course enables students to gain a more extensive vocabulary and study complex grammatical constructions.  They also study Greek and Roman mythology and examine excerpts from the writings of Julius Caesar.  At the end of the course, the students translate four poems from Ovid’s Metamorphoses and learn how to scan Latin poetry.}\\

\noindent\textbf{Latin III} \hfill Crowe

\noindent Year - 1 Credit

\vspace{1mm}\emph{In this course, students study Cicero and Ovid in depth.  Students are also introduced to other great Roman writers: Catullus, Horace, Livy, Martial, and Vergil.}\\

\noindent\textbf{Latin IV} \hfill Crowe

\noindent Year - 1 Credit

\vspace{1mm}\emph{This course is survey of Roman authors featuring readings from Caesar, Catullus, Cicero, and Vergil. Students are exposed to a variety of ancient literary genres and study scansion of the dactylic hexameter in depth. In the spring, students take either the Advanced Prose or Advanced Poetry level of the National Latin Exam.}\\

\noindent\textbf{AP Latin} \hfill Crowe

\noindent Year - 1 Credit

\vspace{1mm}\emph{AP Latin is for fourth-year students who will read selections from Virgil's epic The Aeneid and Julius Caesar's De bello Gallico. Students have the option of taking the AP Latin exam in May.}\\

\noindent\textbf{Spanish I} \hfill TBD

\noindent Year - 1 Credit

\vspace{1mm}\emph{The goal of Spanish 1 is to ensure an understanding and confident use of the most frequent of real, everyday, Spanish words and structures at a novice-high level.  To that end, we will focus on the top 100-200 frequently used Spanish words and structures as well as days of the week, months of the year, numbers, seasons and basic weather terms, basic colors, and sequencing and storytelling terms.  Furthermore, students will be exposed to a wide variety of vocabulary, grammar, and cultural topics through extensive reading incorporated in the curriculum.}\\

\noindent\textbf{Spanish II} \hfill TBD

\noindent Year - 1 Credit

\vspace{1mm}\emph{The goal of Spanish 2 is to ensure an understanding and confident use of the most frequent of real, everyday, Spanish words and structures at an intermediate-low level by reviewing the top 100-200 most frequently used Spanish words and structures and then expanding that list to the top 200-300.  Furthermore, students will be exposed to a wide variety of vocabulary, grammar, and cultural topics through extensive reading incorporated in the curriculum.}\\

\noindent\textbf{Spanish III} \hfill TBD

\noindent Year - 1 Credit

\vspace{1mm}\emph{The goal of Spanish 3 is to ensure an understanding and confident use of the most frequent of real, everyday, Spanish words and structures at an intermediate-mid by reviewing the top 200-300 most frequently used Spanish words and structures and then expanding that list to the top 300-400.  Furthermore, students will be exposed to a wide variety of vocabulary, grammar, and cultural topics through extensive reading incorporated in the curriculum.}\\

\noindent\textbf{Spanish IV} \hfill TBD

\noindent Year - 1 Credit

\vspace{1mm}\emph{The goal of Spanish 4 is to ensure an understanding and confident use of the most frequent of real, everyday, Spanish words and structures at an intermediate-high level by reviewing the top 300-400 most frequently used Spanish words and structures and then expanding that list to the top 400-500.  Furthermore, students will be exposed to a wide variety of vocabulary, grammar, and cultural topics through extensive reading incorporated in the curriculum.}\\

\noindent\textbf{AP Spanish Language \& Culture} \hfill TBD

\noindent Year - 1 Credit

\vspace{1mm}\emph{This course is designed to polish the skills that students have acquired throughout their years of study. In addition to independent review and intensive practice, the course emphasizes a thorough knowledge and understanding of grammar; consolidation of a broad range of sophisticated vocabulary and idiomatic expressions; ability to read and understand literary selections in the original; ability to write essays that balance treatment of content with linguistic control; effective aural comprehension and spoken communication; and evolving understanding of the target cultures. Students who complete this course are prepared to take the Advanced Placement Spanish Language examination.}\\

\noindent\textbf{Adv Spanish Through Film and Literature} \hfill Wald

\noindent Year - 1 Credit

\vspace{1mm}\emph{The goal of Advanced Spanish through Film and Literature is for students to experience and discuss films and readings, communicating at an intermediate-high level of proficiency or higher in Spanish.  Students will be exposed to a wide variety of vocabulary and grammatical structures as well as cultural concepts through the films and extensive reading incorporated in the curriculum.  The course is conducted exclusively in Spanish. }\\

\subsection{Past Courses}
\noindent\textbf{The Works of Marguerite Duras}  - 0.5 Credits

\vspace{3mm}\noindent\textbf{Conversational Spanish}  - 1 Credit

\vspace{3mm}\noindent\textbf{AP Chinese}  - 1 Credit

\vspace{3mm}\noindent\textbf{AP Spanish Literature \& Culture}  - 1 Credit

\vspace{3mm}\noindent\textbf{Creative Writing in Spanish \& English}  - 1 Credit

%\vspace{3mm}\noindent\textbf{Adv Spanish Through Film and Literature}  - 0.5 Credits

\vspace{3mm}\noindent\textbf{20th Century Music and Poetry in Latin and South America}  - 0.5 Credits

\vspace{3mm}\noindent\textbf{Advanced Spanish Linguistics}  - 0.5 Credits

\vspace{3mm}\noindent\textbf{French VI}  - 1 Credit

\vspace{3mm}\noindent\textbf{Intro to Ancient Greek}  - 1 Credit

\vspace{3mm}